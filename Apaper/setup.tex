\documentclass{beamer}
\usepackage{../typesetting/styles/slide-zh}

% Set CJK fonts to local Noto fonts
\setCJKmainfont{Noto Serif CJK SC}
\setCJKsansfont{Noto Sans CJK SC}
\setCJKmonofont{Noto Sans Mono CJK SC}

% Document information
\title{\red{APaper}: \blue{AI} + Paper = \green{研究神器}}
\subtitle{基于MCP服务器的智能论文助手}
\author{isomo}
% \institute{机构名称}
\date{\today}

\begin{document}

% Title frame
\begin{frame}
  \titlepage
\end{frame}

% Outline frame
\begin{frame}{大纲}
  \tableofcontents
\end{frame}

% Section 1: What is APaper?
\section{什么是APaper?}
\begin{frame}{APaper = \red{AI} + Paper}
  \begin{itemize}
    \item \blue{Auto + Paper}: 自动化论文搜索和管理
    \item \green{AI + Paper}: 智能化论文分析和整理
    \item 基于All-in-MCP服务器架构
    \item 让研究变得更轻松、更有趣!
  \end{itemize}
\end{frame}

% Section 2: Simple Demo
\section{简单演示}
\begin{frame}{第一步:搜索论文主题}
  \begin{block}{演示场景}
    想研究"密码学"相关论文怎么办?
  \end{block}
  
  \pause
  
  \begin{itemize}
    \item \red{IACR ePrint Archive}: 密码学顶级预印本库
    \item \blue{Google Scholar}: 跨学科论文搜索
    \item \green{CryptoBib}: 密码学专业文献库
  \end{itemize}
\end{frame}

\begin{frame}{第二步:一键收集所有相关论文}
  \begin{center}
    \huge Search → Read → Summarize
  \end{center}
  
  \begin{itemize}
    \item 自动下载PDF文件
    \item 提取论文元数据(作者、日期、摘要)
    \item 生成BibTeX引用格式
    \item 智能分类和整理
  \end{itemize}
\end{frame}

% Section 3: Configuration
\section{配置教程}
\begin{frame}{如何配置APaper?}
  \begin{block}{前置要求}
    \begin{itemize}
      \item Python 3.12+ 
      \item UV包管理器
    \end{itemize}
  \end{block}
  
  \begin{block}{安装方式}
    \begin{itemize}
      \item \texttt{pip install all-in-mcp} (推荐)
      \item 或从源码安装: \texttt{git clone + uv sync}
    \end{itemize}
  \end{block}
\end{frame}

\begin{frame}[fragile]{MCP客户端集成}
  \begin{block}{Claude Desktop配置示例}
    \begin{verbatim}
{
  "mcpServers": {
    "all-in-mcp": {
      "command": "uv",
      "args": ["run", "all-in-mcp"],
      "cwd": "/path/to/all-in-mcp"
    }
  }
}
    \end{verbatim}
  \end{block}
\end{frame}

% Section 4: Next Steps
\section{下一步发展}
\begin{frame}{创新点:如何找到论文写作的突破口?}
  \begin{itemize}
    \item \red{智能代码生成}: 根据论文需求自动生成实验代码
    \item \blue{写作计划优化}: AI助手帮你制定研究roadmap
    \item \green{创新点发现}: 通过文献分析找到研究gap
  \end{itemize}
\end{frame}

% Thank you slide
\begin{frame}
  \centering
  \LARGE 谢谢

\end{frame}

\end{document}
