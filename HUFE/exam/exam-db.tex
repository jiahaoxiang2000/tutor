\documentclass{article}
\usepackage{../../typesetting/styles/note-zh}
\usepackage{bookmark}
\usepackage{amsmath}
\usepackage{enumitem}

\title{Exam of Database}
\author{isomo}

\begin{document}

\maketitle

\section{2019}

\subsection{选择题(共30分,每题1.5分)}

\begin{enumerate}
\item 数据库系统的数据独立性体现在( )。

A.不会因为数据存储结构与数据逻辑结构的变化而影响应用程序

B.不会因为数据的变化而影响到应用程序

C.不会因为存储策略的变化而影响存储结构

D.不会因为某些存储结构的变化而影响其他的存储结构

\item 关系R(A,\underline{B})和S(B,C)中分别有10个和15个元组,属性B是R的主码,则R与S进行自然连接计算得到的元组数目的范围是( )。

A.[0,15] B.[10,15] C.[10,25] D.[0,150]

\item 在概念模型中的客观存在并可相互区别的事物称( )。

A.元组 B.实体 C.属性 D.节点

\item 设有关系模式R(A,B,C)和S(B,C,D,E),下列关系代数表达式运算出现错误的是( )。

A.$\pi_B(R) \cup \pi_B(S)$ B.$R \cup S$ C.$R \times S$ D.$\pi_{A,B}(R) \Join \pi_{B,C}(S)$

\item 关系数据模型的三个组成部分中,不包括( )。

A.完整性规则     B.数据结构       C.恢复       D.数据操作

\item 设有关系R和关系S进行下图1所示的运算,则运算结果中含有元组的数目是( )。

A.6 B.7 C.8 D.9

---

\begin{tabular}{ccc|cc}
\multicolumn{3}{c|}{R} & \multicolumn{2}{c}{S} \\
\hline
A & B & C & D & E \\
\hline
1 & 2 & 3 & 5 & 6 \\
4 & 5 & 6 & 7 & 8 \\
7 & 8 & 9 & 9 & 10 \\
\end{tabular}

图1

---

\item 数据库中只存放视图的( )。

A.定义 B.操作 C.结果 D.数据

\item SQL中,下列涉及空值的操作,不正确的是( )。

A.AGE IS NOT NULL B.AGE = NULL

C.AGE IS NULL D.NOT (AGE IS NULL)

\item SQL语言具有两种使用方式,一种是交互式SQL,另一种是( )。

A.提示式 B.嵌入式 C.多用户式 D.解释式

\item 有一个关系:学生(学号,姓名,系别),规定学号的值域是8个数字组成的字符串,这一规则属于( )。

A.实体完整性约束 B.参照完整性约束

C.用户自定义完整性约束 D.关键字完整性约束

\item 若事务T对数据对象A加上了X锁,则( )。

A.只允许T修改A,其他任何事务不能再对A加任何类型的锁

B.只允许T读取和修改A,其他任何事务不能再对A加任何类型的锁

C.只允许T读取A,其他任何事务不能再对A加任何类型的锁

D.只允许T修改A,其他任何事务不能再对A加X锁

\item 下面关于函数依赖的叙述中,不正确的是( )。

A.若X→Y,Y→Z,则X→YZ B.若XY→Z,则Y→Z,X→Z

C.若X→Y,Y→Z,则X→Z D.若X→Y,Z$\subseteq$Y,则X→Z

\item 数据库设计可划分为六个阶段,每个阶段都有自己的设计内容,"为哪些关系,在哪些属性上建什么样的索引"这一设计内容应该属于( )阶段。

A.概念结构设计 B.逻辑结构设计 C.物理结构设计 D.全局结构设计

\item 在SQL语言中,子查询是( )。

A.返回单表中数据子集的查询语言

B.选取多表中字段子集的查询语句

C.选取单表中字段子集的查询语句

D.嵌入到另一个查询语句之中的查询语句

\item 已知关系:厂商(厂商号,厂名),主码为厂商号;产品(产品号,颜色,厂商号),主码为产品号,外码厂商号引用厂商表的主码。假设两个关系已经存在如图2所示元组:

\begin{tabular}{ccc|cc}
\multicolumn{3}{c|}{产品} & \multicolumn{2}{c}{厂商} \\
\hline
产品号 & 颜色 & 厂商号 & 厂商号 & 厂名 \\
\hline
P01 & 红 & C01 & C01 & 宏达 \\
P02 & 黄 & C03 & C02 & 立仁 \\
 &  &  & C03 & 广源 \\
\end{tabular}

图2

若再往产品关系中插入如下元组:

I(P03,红,C02)  II(P01,蓝,C01);

III(P04,白,C04) IV(P05,黑,null);

能够插入的元组是( )。

A.I,II,IV B.I,III C.I,II D.I,IV

\item 事务的一致性是指( )。

A.事务必须是使数据库从一个一致性状态变到另一个一致性状态

B.事务一旦提交,对数据库的改变是永久的

C.一个事务内部的操作及使用的数据对开发的其他事务是隔离的

D.事务中包括的所有操作要么都做,要么都不做

\item DBMS中实现事务持久性的子系统是( )。

A.安全性管理子系统 B.恢复管理子系统

C.并发控制子系统 D.完整性管理子系统

\item 在ER模型中,如果有3个不同的实体型,3个M:N联系,根据ER模型转换为关系模型的规则,转换为关系的数目是( )。

A.4 B.5 C.6 D.7

\item 关系模式R中的属性全是主属性,则R的最高范式必定是( )。

A.1NF B.2NF C.3NF D.BCNF

\item 设事务T1和事务T2对数据库中的数据A进行操作可能有如下几种情况,请问哪一种情况不会发生冲突( )。

A.T1正在写A,T2要读A        B.T1正在写A,T2也要写A

C.T1正在读A,T2要写A        D.T1正在读A,T2也要读A
\end{enumerate}

\subsection{填空题(共10分,每空1分)}

\begin{enumerate}
\item DDL的中文全称是\_\_\_\_\_\_\_\_\_\_\_\_\_。

\item SIX锁的中文全称是\_\_\_\_\_\_\_\_\_\_\_\_\_\_\_\_\_。

\item 数据库系统是指在计算机系统中引入数据库后的系统,一般由\_\_\_\_\_\_、数据库管理系统(及其开发工具)、应用系统和数据库管理员构成。

\item 数据库系统的三级模式结构是内模式、\_\_\_\_\_\_\_\_\_\_\_、外模式。

\item SQL语言用\_\_\_\_\_\_\_\_\_\_\_\_\_(此空填英文单词)语句向用户授予对数据的操作权限。

\item 查询优化是指选择一个高效执行的查询处理策略。查询优化按照优化的层次一般可分为代数优化和\_\_\_\_\_\_\_\_\_\_\_\_\_\_\_\_\_。

\item 五种基本关系代数运算是并、差、\_\_\_\_\_\_\_\_\_\_、选择、\_\_\_\_\_\_\_\_\_\_。

\item 并发操作带来的数据不一致性包括:丢失修改、\_\_\_\_\_\_\_\_\_\_\_\_、读"脏"数据。

\item 数据库系统中诊断死锁的方式一般使用超时法或\_\_\_\_\_\_\_\_\_\_\_\_\_\_\_。
\end{enumerate}

\subsection{简答题(共20分,第1题8分,第2题6分,第3题6分)}

\begin{enumerate}
\item 假设某商业集团数据库中有一关系模式R如下:

R(商店编号,商品编号,数量,部门编号,负责人),如果规定:

\begin{itemize}
\item 每个商店的每种商品只在一个部门销售;
\item 每个商店的每个部门只有一个负责人;
\item 每个商店的每种商品只有一个库存数量。
\end{itemize}

试回答下列问题:

(1) 根据上述规定,写出关系模式R的基本函数依赖;

(2) 找出关系模式R的候选码;

(3) 试问关系模式R最高已经达到第几范式?为什么?

\item 数据库运行的过程中,某个存储了数据的磁盘扇区坏了。针对这类故障,请写出相应的恢复策略与方法(假设你拥有某个时刻T的数据库的海量静态转储副本,并拥有T时刻到故障发生时刻的日志文件副本)。

\item 已知有三个事务的一个调度$R_3(B)R_1(A)W_3(B)R_2(B)R_2(A)W_2(B)R_1(B)W_1(A)$,试问该调度是否是冲突可串行化调度?为什么?
\end{enumerate}

\subsection{综合应用题(共30分)}

某大学举行运动会,要求建立一个简单的数据库系统管理学生的比赛成绩,经过分析得到的ER模型图如图3所示,Student表示学生实体(属性Sno、Sname、Ssex、Sage、Sdept分别表示学生的学号、姓名、性别、年龄、所在系),Sports表示运动项目实体(属性SportNo、SportName、SportUnit分别表示运动项目的编号、名称、项目的计分单位),Student与Sports之间的参与关系用SS表示(联系的属性Grade表示比赛成绩)。各表的结构如表1、表2、表3所示。

\begin{enumerate}
\item 根据题目要求,写出相应的SQL语句。

(1) 写出创建表SS的SQL代码(6分)。

(2) 从表SS中删除学生"张三"的参与比赛项目的记录(假设只有一个"张三") (3分)。

(3) 为SS表添加一条记录,学号为"xh001"的学生参与了编号为"xm001"的运动项目,但还没成绩(3分)。

(4) 查询"计算机"系的学生参加了哪些运动项目,只把运动项目名称列出,去除重复记录(3分)。

(5) 查询各个系的学生的"跳高"项目比赛的平均成绩 (不要求输出比赛项目的计分单位) (3分)。

(6) 建立"计算机"系所有男学生的信息视图JSJ\_M\_Student(3分)。

(7) 回收用户"李明"对Sports表的查询权限(3分)。

\item 用关系代数表达式表达以下查询。

(1) 查询参加"跳高"的学生的姓名(3分)。

(2) 查询参加了所有运动项目的学生姓名(3分)。
\end{enumerate}

\subsection{设计题(10分)}

假设有"教师"、"学生"、"课程"三个实体。一门课程只能有一个教师任课,一个教师可以上多门课程;一个学生可以选修多门课程,一门课程可以由多个学生来选修。已知教师的属性有:工号、姓名、职称,课程的属性有课程号、课程名、学时数,学生的属性有学号、姓名、性别、年龄。根据上述描述,解答下列问题:

(1)设计并画出E-R图,要求标注连通词(4分);

(2)将E-R图转化为关系模型,并指出各关系的主码和外码(6分)。

\section{2020}

\subsection{选择题(15空,每空2分,共30分)}
\begin{enumerate}
\item 数据库系统中,长期存储在计算机内有结构的数据集合称为( )。

A. 数据库管理系统 B.数据模型 C.数据结构 D.数据库

\item 从E-R模型向关系模型转换时,一个M:N联系转换为关系模式时,该关系模式的关键字为( )。

A.N端实体的关键字

B.M端实体的关键字

C.M端实体的关键字和N端实体的关键字组合

D.重新选取其他属性

\item 当对某一表进行诸如哪些操作时,SQL Server 不会自动执行触发器所定义的SQL 语句。( )

A. Insert B.Update C.Delete D.Select

\item SQL语言用于实现增加,删除,修改使用的是哪种功能( )

A.关系规范化、数据操纵、数据控制 B.数据定义、数据操纵、数据控制

C.数据定义、关系规范化、数据控制 D.数据定义、关系规范化、数据操纵

\item 数据库系统的数据独立性是指( )

A.不会因为存储策略的变化而影响存储结构

B.不会因为数据的变化而影响应用程序

C.不会因为系统数据存储结构与数据逻辑结构的变化而影响应用程序

D.不会因为某些存储结构的变化而影响其他的存储结构

\item 表的Check约束是( )的有效性检验规则。

A.实体完整性 B.参照完整性

C.用户自定义完整性 D.唯一完整性

\item 一个工厂可以生产多个零件,一个零件可以在多个工厂生产,零件和工厂之间的关系( )。

A.1:N B. 1:M

C. M:N D. 1:M:N

\item 检索所有比"王华"年龄大的学生姓名、年龄和性别。正确的SELECT语句是( )  。

A.SELECT SN,AGE,SEX FROM S WHERE AGE>(SELECT AGE FROM S

WHERE SN="王华")

B.SELECT SN,AGE,SEX FROM S WHERE SN="王华"

C.SELECT SN,AGE,SEX FROM S WHERE AGE>(SELECT AGE WHERE SN="王华")

D.SELECT SN,AGE,SEX FROM S WHERE AGE>王华.AGE

\item 信息世界中的术语元组,与之对应的数据库术语为( )。

A.文件 B.数据库 C.字段 D.记录

\item 关系数据库管理系统应能实现的专门关系运算包括( )。

A.排序、索引、统计 B.选择、投影、连接

C.关联、更新、排序 D.显示、打印、制表

\item 在数据库中建立索引的目的是( )。

A.节省存储空间 B.提高查询速度

C.提高查询和更新速度 D.提高更新速度

\item 关系模型中,一个关键字是( )。

A.可由多个任意属性组成 B.至多由一个属性组成

C.可由一个或多个其值能唯一标识该关系模式中任何元组的属性组成

D.以上都不是

\item 给定关系R(A,B,C,D)和关系S(B,C,E),对其进行自然连接运算后的属性列为( )个。

A.7 B.4 C.2 D.5

\item 下列叙述错误的是

A.视图是一个虚表,是从一个或几个基本表导出的表

B.数据库中既存放视图的定义,又存放视图对应的数据

C.可以在视图之上再定义新的视图

D.基本表中的数据发生变化,视图中查寻得出的数据也就改变了

\item ( )是数据库系统中各种描述信息和控制信息的集合。

A.数据字典 B.数据流 C.数据结构 D.数据
\end{enumerate}

\subsection{填空题(5题, 每题2分,共10分)}
\begin{enumerate}
\item 实体之间的联系可抽象为三类,它们是 一对一、 和 。

\item 数据模型是由数据结构 、 和 三部分组成的。

\item 数据库发展经历了哪些阶段 , , 。

\item 数据完整性约束条件分为哪三类 , , 。

\item 若一个关系模式的非主属性既不部分依赖于主码,也不对主码具有传递依赖。则该关系模式属于 。
\end{enumerate}

\subsection{判断题(10题,每题2分,共20分)}
\begin{enumerate}
\item 数据库的字符串数据类型有varchar和nvarchar两种。 ( )

\item 数据库的安全性是指数据库的正确性和有效性。 ( )

\item 选择运算是针对表进行的垂直分割运算。 ( )

\item 关系数据模型是数据库系统中最早出现的数据模型。 ( )

\item 数据独立性是指应用程序数据的逻辑和物理独立性是相互独立的 ( )

\item 部门(部门号,部门名,部门成员,部门总经理)满足1NF要求。 ( )

\item 层次数据模型的结构是树形结构。 ( )

\item 在关系数据库中,描述全局数据逻辑结构的是外模式。 ( )

\item 任何二维表都可以转换为关系模式。 ( )

\item 记录是对实体特征的描述 ( )
\end{enumerate}

\subsection{简答题(2题,每小题5分,共10分)}

\begin{enumerate}
\item 什么是数据库?数据库有哪些特点?

\item SQL指的是什么?它有哪几个动词?
\end{enumerate}

\subsection{综合应用题(共30分)}

\subsubsection{SQL语句题(20分)}

学生表Student(Sno,XM,XB,CSRQ,)

其中Sno学号,XM姓名、XB性别、CSRQ出生日期、JXJ奖学金、GYZZ公寓住址

\begin{enumerate}
\item 创建学生表(Student),奖学金为小数位(7.2),出生日期为时间日期类型,并\textbf{要求指出主码}(4分)

\item 插入一条学生记录,'2012','李娟','女','1999-10-26',500(4分)

\item 修改表:在表中插入一个家庭住址(Addr)列(3分)

\item 数据更新:把女生的公寓住址改为E栋102(3分)

\item 查询所有女生小于等于500元奖学金的记录(3分)

\item 查询和和李娟住同一个地址的学生学号和姓名(3分)
\end{enumerate}

\subsubsection{应用题(10分)}

将下图M:N联系转换为关系模式,并注明主码。

零件 $\stackrel{\text{M}}{\longleftrightarrow}$ 供应 $\stackrel{\text{N}}{\longleftrightarrow}$ 供应商

% % 添加参考文献
% \bibliographystyle{plain}
% \bibliography{references}

\end{document}
