\documentclass{../../note}

\usepackage{amsthm}
\usepackage{pgfplots}
\pgfplotsset{compat=1.18}
\newtheorem{example}{Example}
\usepackage{xcolor} % For colored text
\usepackage{tikz}
\usetikzlibrary{shapes,arrows,positioning,fit,calc,matrix,decorations.pathreplacing}
\usepackage{algorithm}
\usepackage{algpseudocode}
\usepackage{listings}
\lstset{
  basicstyle=\ttfamily\small,
  keywordstyle=\color{blue},
  commentstyle=\color{green!60!black},
  stringstyle=\color{purple},
  numbers=left,
  numberstyle=\tiny,
  numbersep=5pt,
  breaklines=true,
  frame=single,
}

\title{数据库 01}
\author{isomo}

\begin{document}

\maketitle

考纲内容:

\begin{itemize}

  \item \textbf{绪论}\\
    数据库的4个基本概念,数据管理技术的产生和发展,数据建模、概念模型和数据模型的三要素,数据库系统的三级模式结构,数据库的两级映像与数据独立性,数据库系统的组成。

  \item \textbf{关系模型}\\
    关系模型的数据结构及形式化定义,关系操作,关系完整性,关系代数(传统的集合运算、专门的关系运算)。

  \item \textbf{关系数据库标准语言SQL}\\
    数据定义、数据查询、数据更新、空值处理、视图。ss

  \item \textbf{数据库安全性}\\
    数据库安全性概述,数据库安全性控制。

\end{itemize}

\section{绪论}

\subsection{数据库的基本概念}

数据库系统的四个基本概念:

\begin{enumerate}
  \item \textcolor{red}{数据 (Data)}:描述事物的符号记录,是数据库中存储的基本对象。

  \item \textcolor{red}{数据库 (Database, DB)}:长期存储在计算机内、有组织的、可共享的大量数据的集合。数据库中的数据具有如下特点:
    \begin{itemize}
      \item 永久存储
      \item 有组织
      \item 可共享
    \end{itemize}

  \item \textcolor{red}{数据库管理系统 (Database Management System, DBMS)}:位于用户与操作系统之间的一层数据管理软件,是基础软件,是一个大型复杂的软件系统。

  \item \textcolor{red}{数据库系统 (Database System, DBS)}:由数据库、数据库管理系统(及其应用开发工具)、应用程序和数据库管理员(DBA)组成的存储、管理、处理和维护数据的系统。
\end{enumerate}

\subsection{数据管理技术的产生和发展}

数据管理技术的发展经历了三个阶段:

\begin{enumerate}
  \item \textbf{人工管理阶段}(20世纪50年代中期以前)
    \begin{itemize}
      \item 数据不保存
      \item 应用程序管理数据
      \item 缺点:数据不可共享、数据冗余度大、数据不一致性
    \end{itemize}

  \item \textbf{文件系统阶段}(20世纪50年代中期至60年代中期)
    \begin{itemize}
      \item 数据可长期保存
      \item 文件系统实现对数据的管理
      \item 缺点:数据共享性差、数据冗余大、数据独立性差
    \end{itemize}

  \item \textbf{数据库系统阶段}(20世纪60年代末至今)
    \begin{itemize}
      \item 数据结构化
      \item 数据共享性高,冗余度低
      \item 数据独立性高
      \item 由DBMS统一管理和控制
    \end{itemize}
\end{enumerate}

\subsection{数据建模与数据模型}

\subsubsection{数据建模}

\textcolor{red}{数据建模}是抽象、表示和处理现实世界中数据的方法和过程。

\subsubsection{概念模型}

\textcolor{red}{概念模型}是按用户的观点来对数据和信息建模,主要用于数据库设计。

最常用的概念模型是\textcolor{blue}{实体-联系模型 (E-R模型)},它由下列要素组成:
\begin{itemize}
  \item 实体(Entity):客观存在并可相互区别的事物
  \item 属性(Attribute):实体所具有的某一特性
  \item 联系(Relationship):实体之间的关联
\end{itemize}

\begin{figure}[h]
  \centering
  \begin{tikzpicture}
    % 绘制实体
    \node[draw, rectangle, minimum width=2cm, minimum height=1cm] (student) {学生};
    \node[draw, rectangle, minimum width=2cm, minimum height=1cm, right=4cm of student] (course) {课程};

    % 绘制联系
    \node[draw, diamond, aspect=2, minimum width=2cm, minimum height=1cm] (sc) at ($(student)!0.5!(course)$) {选修};

    % 连接线
    \draw (student) -- (sc);
    \draw (sc) -- (course);

    % 标注关系
    \node[above] at ($(student)!0.25!(sc)$) {1:n};
    \node[above] at ($(course)!0.25!(sc)$) {1:n};
  \end{tikzpicture}
  \caption{E-R图示例:学生-课程关系}
\end{figure}

\subsubsection{数据模型的三要素}

\textcolor{red}{数据模型}是对现实世界数据特征的抽象,由三部分组成:

\begin{enumerate}
  \item \textbf{数据结构}:描述数据库的组成对象及对象间的联系
  \item \textbf{数据操作}:对数据库中各种对象实例允许执行的操作及操作规则
  \item \textbf{数据约束}:保证数据库中数据满足特定语义规则的条件
\end{enumerate}

按照抽象级别,数据模型可分为:
\begin{itemize}
  \item \textbf{概念模型}:面向用户,如E-R模型
  \item \textbf{逻辑模型}:面向DBMS,如层次模型、网状模型、关系模型、面向对象模型等
  \item \textbf{物理模型}:面向存储,描述数据在存储介质上的实际组织方式
\end{itemize}

\subsection{数据库系统的三级模式结构}

ANSI/SPARC提出的三级模式结构包括:

\begin{enumerate}
  \item \textcolor{red}{外模式 (External Schema)}:也称为用户模式,是用户与数据库系统的接口,由若干外部视图组成。

  \item \textcolor{red}{模式 (Schema)}:也称为概念模式,是数据库中全体数据的逻辑结构和特征的描述,是所有用户的公共数据视图。

  \item \textcolor{red}{内模式 (Internal Schema)}:也称为存储模式,是数据物理结构和存储方式的描述,是数据在存储介质上的表示方式和存取方法。
\end{enumerate}

\colorbox{yellow}{注:此处三种模式,其实对应用户,软件开发者和数据库开发人员。}

\begin{figure}[h]
  \centering
  \begin{tikzpicture}
    % 创建三级模式框
    \node[draw, minimum width=10cm, minimum height=1.2cm] (ext) {外模式1 \hspace{2cm} 外模式2 \hspace{2cm} 外模式3};
    \node[draw, minimum width=10cm, minimum height=1.2cm, below=1cm of ext] (schema) {模式(概念模式)};
    \node[draw, minimum width=10cm, minimum height=1.2cm, below=1cm of schema] (int) {内模式(存储模式)};
    \node[draw, minimum width=10cm, minimum height=1.2cm, below=1cm of int] (db) {数据库};

    % 添加映射
    \draw[<->] (ext.south) -- (schema.north) node[midway, right] {外模式/模式映射};
    \draw[<->] (schema.south) -- (int.north) node[midway, right] {模式/内模式映射};
    \draw[<->] (int.south) -- (db.north);
  \end{tikzpicture}
  \caption{数据库系统的三级模式结构}
\end{figure}

\subsection{数据库的两级映像与数据独立性}

\subsubsection{两级映像}

\begin{enumerate}
  \item \textcolor{blue}{外模式/模式映像}:定义外模式与模式之间的对应关系,当模式改变时,对应的外模式/模式映像也需要改变。

  \item \textcolor{blue}{模式/内模式映像}:定义模式与内模式之间的对应关系,当内模式改变时,对应的模式/内模式映像也需要改变。
\end{enumerate}

\subsubsection{数据独立性}

\begin{enumerate}
  \item \textcolor{red}{物理数据独立性}:当数据库的内模式改变时,只需要修改模式/内模式映像,使模式保持不变,应用程序不受影响。

  \item \textcolor{red}{逻辑数据独立性}:当数据库的模式改变时,只需要修改外模式/模式映像,使外模式保持不变,应用程序不受影响。
\end{enumerate}

数据独立性是数据库系统的重要特征,它保证了应用程序和数据库结构的相对独立,从而提高了数据库系统的可维护性和扩展性。

\subsection{数据库系统的组成}

数据库系统由以下几部分组成:

\begin{enumerate}
  \item \textbf{硬件平台}:计算机、存储设备和网络设备等

  \item \textbf{数据库}:存储在计算机中的数据集合

  \item \textbf{DBMS}:管理数据库的软件

  \item \textbf{应用程序}:为用户提供操作界面的程序

  \item \textbf{数据库管理员(DBA)}:负责数据库的规划、设计、维护和管理

  \item \textbf{用户}:使用数据库的人,包括最终用户、应用程序员和DBA
\end{enumerate}

\begin{figure}[h]
  \centering
  \begin{tikzpicture}
    % 创建一个矩阵节点以对齐所有元素
    \matrix [column sep=0.5cm, row sep=0.4cm] {
      % 用户层
      \node[draw, minimum width=2cm, minimum height=0.8cm] (eu) {最终用户}; &
      \node[draw, minimum width=2cm, minimum height=0.8cm] (ap) {应用程序员}; &
      \node[draw, minimum width=2cm, minimum height=0.8cm] (dba) {DBA}; \\

      % 应用层
      \node[draw, minimum width=7cm, minimum height=0.8cm] (apps) {应用程序}; \\

      % DBMS层
      \node[draw, minimum width=7cm, minimum height=1.2cm] (dbms) {
        数据库管理系统(DBMS)
      }; \\

      % 数据层
      \node[draw, minimum width=7cm, minimum height=0.8cm] (db) {数据库}; \\

      % 硬件层
      \node[draw, minimum width=7cm, minimum height=0.8cm] (hw) {硬件}; \\
    };

    % 连接各层
    \draw[->] (eu) -- (apps);
    \draw[->] (ap) -- (apps);
    \draw[->] (dba) -- (dbms);
    \draw[->] (apps) -- (dbms);
    \draw[<->] (dbms) -- (db);
    \draw[<->] (db) -- (hw);
  \end{tikzpicture}
  \caption{数据库系统的组成}
\end{figure}

\subsection{小结}

\begin{itemize}
  \item 数据库系统的四个基本概念:数据、数据库、数据库管理系统和数据库系统
  \item 数据管理技术的发展经历了人工管理、文件系统和数据库系统三个阶段
  \item 数据模型由数据结构、数据操作和数据约束三要素组成
  \item 数据库系统采用三级模式结构:外模式、模式和内模式
  \item 数据库的两级映像支持物理数据独立性和逻辑数据独立性
  \item 数据库系统由硬件、软件、数据、人员等组成部分构成
\end{itemize}

\section{关系模型}

\subsection{关系模型的数据结构}

关系模型是目前最重要的数据库模型,由E.F.Codd于1970年首先提出。关系模型的基本数据结构非常简单,就是关系,即二维表格结构。

\subsubsection{关系的形式化定义}

设有$n$个域$D_1, D_2, \ldots, D_n$,它们的笛卡尔积为:
\begin{align}
  D_1 \times D_2 \times \ldots \times D_n = \{(d_1, d_2, \ldots, d_n) | d_i \in D_i, i = 1, 2, \ldots, n\}
\end{align}

\textcolor{red}{关系 (Relation)}是笛卡尔积$D_1 \times D_2 \times \ldots \times D_n$的子集,表示为$R(A_1:D_1, A_2:D_2, \ldots, A_n:D_n)$,其中:
\begin{itemize}
  \item $R$是关系名
  \item $A_i$是属性名
  \item $D_i$是域(属性的取值范围)
  \item $n$是关系的目或度(Degree),表示关系的属性个数
  \item 关系中的每个元组(Tuple)对应表中的一行
  \item 关系模式(Relation Schema):$R(A_1, A_2, \ldots, A_n)$
\end{itemize}

\subsubsection{关系的性质}

\begin{enumerate}
  \item \textcolor{blue}{列是同质的}:每一列中的数据来自同一个域,是同一类型的数据
  \item \textcolor{blue}{不同的列可来自同一个域}:不同属性可对应相同的域
  \item \textcolor{blue}{列的顺序无关紧要}:列的次序可以任意交换
  \item \textcolor{blue}{行的顺序无关紧要}:行的次序可以任意交换
  \item \textcolor{blue}{行列确定唯一的值}:给定行号和列名后,表中的值唯一确定
  \item \textcolor{blue}{不允许表中有重复的行(元组)}:任意两个元组在至少一个属性上取值不同
  \item \textcolor{blue}{每个分量必须是不可分的数据项}:不允许表中的表(非规范化)
\end{enumerate}

\begin{example}
  某大学学生关系Student的一个实例:

  \begin{center}
    \begin{tabular}{|c|c|c|c|}
      \hline
      \textbf{Sno} & \textbf{Sname} & \textbf{Ssex} & \textbf{Sage} \\
      \hline
      201901 & 李勇 & 男 & 20 \\
      \hline
      201902 & 刘晨 & 女 & 19 \\
      \hline
      201903 & 王敏 & 女 & 18 \\
      \hline
    \end{tabular}
  \end{center}

  关系模式:Student(Sno, Sname, Ssex, Sage)
\end{example}

\subsubsection{关系模型中的基本概念}

\begin{enumerate}
  \item \textcolor{red}{候选键(Candidate Key)}:能唯一标识关系中元组的最小属性集合。
  \item \textcolor{red}{主键(Primary Key)}:从候选键中选定的一个,用于唯一标识关系中的元组。
  \item \textcolor{red}{外键(Foreign Key)}:关系R的一个属性(或属性集),它不是R的主键,但是它在另一个关系S中是主键。
  \item \textcolor{red}{主属性(Prime Attribute)}:包含在任何一个候选键中的属性。
  \item \textcolor{red}{非主属性(Non-prime Attribute)}:不包含在任何候选键中的属性。
\end{enumerate}

\subsection{关系操作}

关系模型的操作主要分为查询和更新两类。

\subsubsection{查询操作}

查询操作是关系数据库中最基本的操作,主要包括:
\begin{itemize}
  \item 选择(Selection):从关系中选取满足条件的元组
  \item 投影(Projection):从关系中选取指定的列
  \item 连接(Join):将两个关系按照共同属性组合成一个关系
  \item 除法(Division):A ÷ B,求A中满足B中所有条件的元组
  \item 并(Union):两个关系的并集
  \item 差(Difference):两个关系的差集
  \item 交(Intersection):两个关系的交集
  \item 笛卡尔积(Cartesian Product):两个关系的所有可能组合
\end{itemize}

\subsubsection{更新操作}

更新操作包括:
\begin{itemize}
  \item 插入(Insert):向关系中添加元组
  \item 删除(Delete):从关系中删除元组
  \item 修改(Update):修改关系中的元组
\end{itemize}

\subsection{关系完整性}

关系模型中的完整性约束是保证数据库中数据正确性、有效性和相容性的规则,主要包括:

\subsubsection{实体完整性(Entity Integrity)}

\textcolor{red}{实体完整性规则}:关系的主键属性值不能为空(NULL)。

这保证了每个实体(即关系中的每一行)都能被唯一标识。

\subsubsection{参照完整性(Referential Integrity)}

\textcolor{red}{参照完整性规则}:如果关系R的外键F是关系S的主键,则关系R中每个元组在F上的取值必须是:
\begin{itemize}
  \item 要么等于关系S中某个元组的主键值
  \item 要么为空值(如果允许外键取空值)
\end{itemize}

\subsubsection{用户定义的完整性(User-defined Integrity)}

用户定义的完整性是针对具体应用的约束条件,例如:
\begin{itemize}
  \item 属性值的范围约束(例如年龄必须大于0且小于120)
  \item 属性间的相互约束(例如入职日期必须晚于出生日期)
\end{itemize}

\subsection{关系代数}

关系代数是一种抽象的查询语言,它用对关系的运算来表达查询。

\subsubsection{传统的集合运算}

\begin{enumerate}
  \item \textcolor{blue}{并(Union)}:$R \cup S = \{t | t \in R \vee t \in S\}$

  \item \textcolor{blue}{差(Difference)}:$R - S = \{t | t \in R \wedge t \notin S\}$

  \item \textcolor{blue}{交(Intersection)}:$R \cap S = \{t | t \in R \wedge t \in S\}$

  \item \textcolor{blue}{笛卡尔积(Cartesian Product)}:$R \times S = \{(r, s) | r \in R \wedge s \in S\}$
\end{enumerate}

需要注意的是,进行并、差、交运算的两个关系必须是\textcolor{red}{同元(union-compatible)}的,即它们必须具有相同的目(属性数)且对应的属性来自相同的域。

\begin{figure}[h]
  \centering
  \begin{tikzpicture}
    % 第一个集合
    \draw (0,0) circle (1);
    \node at (-0.5, 0) {$R$};

    % 第二个集合
    \draw (1.5,0) circle (1);
    \node at (2, 0) {$S$};

    % 标记区域
    \node at (0.75, -1.5) {$R \cup S$: 两个圆的全部区域};
    \node at (0.75, -2) {$R \cap S$: 两个圆的交叉区域};
    \node at (0.75, -2.5) {$R - S$: 仅在$R$中的区域};
  \end{tikzpicture}
  \caption{集合运算示意图}
\end{figure}

\subsubsection{专门的关系运算}

\begin{enumerate}
  \item \textcolor{blue}{选择(Selection)}:$\sigma_F(R) = \{t | t \in R \wedge F(t) = \textrm{true}\}$

    选择操作是从关系R中选取满足给定条件F的元组。

  \item \textcolor{blue}{投影(Projection)}:$\Pi_A(R) = \{t[A] | t \in R\}$

    投影操作是从关系R中选取指定的属性A组成新的关系。

  \item \textcolor{blue}{连接(Join)}:
    \begin{itemize}
      \item 自然连接(Natural Join):$R \bowtie S = \{rs | r \in R \wedge s \in S \wedge r[A] = s[A]\}$,其中A是R和S共有的属性集合。
      \item $\theta$-连接(Theta Join):$R \bowtie_\theta S = \{rs | r \in R \wedge s \in S \wedge \theta(r, s)\}$
      \item 外连接(Outer Join):保留在连接中无匹配的元组,缺少的属性值用NULL填充。
    \end{itemize}

  \item \textcolor{blue}{除法(Division)}:$R \div S = \{t[X] | t \in R \wedge \forall s \in S, ts \in R\}$

    除法操作用于回答"对于S中的所有值,找出R中与之相关的所有值"类型的查询。
\end{enumerate}

\begin{figure}[h]
  \centering
  \begin{tikzpicture}
    % 关系R
    \node[draw, rectangle, minimum width=3cm, minimum height=2cm] (R) at (0,0) {\shortstack{关系R\\(A, B, C)}};

    % 关系S
    \node[draw, rectangle, minimum width=3cm, minimum height=2cm] (S) at (5,0) {\shortstack{关系S\\(C, D, E)}};

    % 连接操作
    \node[draw, rectangle, minimum width=3cm, minimum height=2cm] (join) at (2.5,-3) {\shortstack{R $\bowtie$ S\\(A, B, C, D, E)}};

    % 箭头
    \draw[->] (R) -- (join);
    \draw[->] (S) -- (join);

    % 注释
    \node at (2.5,-4.5) {自然连接:基于共同属性C};
  \end{tikzpicture}
  \caption{自然连接示例}
\end{figure}

\begin{example}
  考虑以下两个关系:

  学生关系Student(Sno, Sname, Sage, Sdept)\\
  选课关系SC(Sno, Cno, Grade)

  以下关系代数表达式表示"查询所有选修了'C1'课程的学生姓名":

  $\Pi_\text{Sname}(\sigma_\text{Cno='C1'}(\text{Student} \bowtie \text{SC}))$

  执行过程:
  \begin{enumerate}
    \item 首先,Student和SC进行自然连接(按Sno属性)
    \item 然后,选择Cno='C1'的元组
    \item 最后,投影出Sname属性
  \end{enumerate}
\end{example}

\subsection{扩展的关系代数操作}

除了基本的关系代数操作外,还有一些扩展操作:

\begin{enumerate}
  \item \textcolor{blue}{广义投影(Generalized Projection)}:允许在投影列表中包含计算表达式
  \item \textcolor{blue}{聚集(Aggregation)}:包括COUNT、SUM、AVG、MAX、MIN等聚合函数
  \item \textcolor{blue}{外连接(Outer Join)}:保留在连接中无匹配的元组
  \item \textcolor{blue}{半连接(Semi-join)}:$R \ltimes S = \Pi_R(R \bowtie S)$
\end{enumerate}

\subsection{小结}

\begin{itemize}
  \item 关系模型以简单的二维表格形式表示数据,具有形式化的数学基础
  \item 关系的基本特性包括列的同质性、行列无序性、元组的唯一性等
  \item 关系完整性约束包括实体完整性、参照完整性和用户定义的完整性
  \item 关系代数提供了一套形式化的操作,分为传统的集合运算和专门的关系运算
  \item 关系代数是关系数据库查询语言的理论基础,SQL语言实现了关系代数的大部分功能
\end{itemize}

\end{document}
