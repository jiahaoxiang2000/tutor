\documentclass{../../note}

\usepackage{amsthm}
\usepackage{pgfplots}
\pgfplotsset{compat=1.18}
\newtheorem{example}{Example}
\usepackage{xcolor} % For colored text
\usepackage{tikz}
\usetikzlibrary{shapes,arrows,positioning,fit,calc,matrix}
\usepackage{algorithm}
\usepackage{algpseudocode}
\usepackage{listings}
\lstset{
  basicstyle=\ttfamily\small,
  keywordstyle=\color{blue},
  commentstyle=\color{green!60!black},
  stringstyle=\color{purple},
  numbers=left,
  numberstyle=\tiny,
  numbersep=5pt,
  breaklines=true,
  frame=single,
}

\title{数据结构 02}
\author{isomo}

\begin{document}

\maketitle

考纲内容:

\begin{itemize}

  \item \textbf{栈和队列: }
    栈的结构特性、基本操作及在顺序存储结构和链式存储结构上基本运算的实现,队列的结构特性、基本操作及在
    顺序存储结构和链式存储结构上基本运算的实现,栈和队列的基本应用。

  \item \textbf{数组和广义表: }
    数组的基本概念和存储结构,广义表的定义和存储结构。
\end{itemize}

\section{栈和队列}

\subsection{栈}

\subsubsection{栈的基本概念}

\textcolor{red}{栈(Stack)的定义}:栈是一种线性表,但限定在表的一端(称为栈顶,top)进行插入和删除。另一端称为栈底(bottom)。

栈的特点是:\textbf{后进先出}(Last In First Out, LIFO)。

\begin{figure}[h]
  \centering
  \begin{tikzpicture}
    \draw (0,0) -- (0,3) -- (2,3) -- (2,0);
    \draw (0,0) -- (2,0);
    \node at (1,-0.3) {栈底 (bottom)};
    \node at (1,3.3) {栈顶 (top)};
    \draw[fill=lightgray] (0,0) rectangle (2,0.5);
    \draw[fill=lightgray] (0,0.5) rectangle (2,1);
    \draw[fill=lightgray] (0,1) rectangle (2,1.5);
    \node at (1,0.25) {元素1};
    \node at (1,0.75) {元素2};
    \node at (1,1.25) {元素3};
    \draw[->, thick] (3,2) -- (1.9,1.7);
    \node at (4,2) {进栈/出栈操作};
  \end{tikzpicture}
  \caption{栈的结构示意图}
\end{figure}

\subsubsection{栈的基本操作}

栈的基本操作包括:

\begin{itemize}
  \item \textcolor{blue}{InitStack(\&S)}:初始化栈,构造一个空栈S
  \item \textcolor{blue}{StackEmpty(S)}:判断栈是否为空
  \item \textcolor{blue}{Push(\&S, x)}:进栈,将元素x压入栈S
  \item \textcolor{blue}{Pop(\&S, \&x)}:出栈,弹出栈顶元素并通过x返回
  \item \textcolor{blue}{GetTop(S, \&x)}:读栈顶元素,通过x返回栈顶元素(不改变栈)
  \item \textcolor{blue}{ClearStack(\&S)}:清空栈
\end{itemize}

\subsubsection{栈的顺序存储结构}

顺序栈通常使用一维数组和一个指向栈顶的指针来实现:

\begin{lstlisting}[language=C]
#define MaxSize 100  // 栈的最大容量
typedef struct {
    ElemType data[MaxSize];  // 存放栈中元素的数组
    int top;                 // 栈顶指针
} SqStack;
\end{lstlisting}

栈顶指针 \texttt{top} 的含义:
\begin{itemize}
  \item 初始化时,\texttt{top = -1}
  \item 进栈操作:先 \texttt{top++},再存入元素
  \item 出栈操作:先取出栈顶元素,再 \texttt{top--}
\end{itemize}

\textcolor{blue}{栈的基本操作实现}:

1. 初始化
\begin{lstlisting}[language=C]
void InitStack(SqStack *S) {
    S->top = -1;
}
\end{lstlisting}

2. 判断栈空
\begin{lstlisting}[language=C]
bool StackEmpty(SqStack S) {
    return S.top == -1;
}
\end{lstlisting}

3. 进栈操作
\begin{lstlisting}[language=C]
bool Push(SqStack *S, ElemType x) {
    if(S->top == MaxSize-1)  // 栈满
        return false;
    S->top++;
    S->data[S->top] = x;
    return true;
}
\end{lstlisting}

4. 出栈操作
\begin{lstlisting}[language=C]
bool Pop(SqStack *S, ElemType *x) {
    if(S->top == -1)  // 栈空
        return false;
    *x = S->data[S->top];
    S->top--;
    return true;
}
\end{lstlisting}

5. 获取栈顶元素
\begin{lstlisting}[language=C]
bool GetTop(SqStack S, ElemType *x) {
    if(S.top == -1)  // 栈空
        return false;
    *x = S.data[S.top];
    return true;
}
\end{lstlisting}

\subsubsection{栈的链式存储结构}

链栈通常使用单链表实现,且规定所有操作都在链表的头部进行:

\begin{lstlisting}[language=C]
typedef struct LinkNode {
    ElemType data;
    struct LinkNode *next;
} LinkNode, *LinkStack;
\end{lstlisting}

\textcolor{blue}{链栈的基本操作实现}:

1. 初始化
\begin{lstlisting}[language=C]
void InitStack(LinkStack *S) {
    *S = NULL;
}
\end{lstlisting}

2. 判断栈空
\begin{lstlisting}[language=C]
bool StackEmpty(LinkStack S) {
    return S == NULL;
}
\end{lstlisting}

3. 进栈操作
\begin{lstlisting}[language=C]
void Push(LinkStack *S, ElemType x) {
    LinkNode *p = (LinkNode *)malloc(sizeof(LinkNode));
    p->data = x;
    p->next = *S;
    *S = p;
}
\end{lstlisting}

4. 出栈操作
\begin{lstlisting}[language=C]
bool Pop(LinkStack *S, ElemType *x) {
    if(*S == NULL)
        return false;
    LinkNode *p = *S;
    *x = p->data;
    *S = p->next;
    free(p);
    return true;
}
\end{lstlisting}

5. 获取栈顶元素
\begin{lstlisting}[language=C]
bool GetTop(LinkStack S, ElemType *x) {
    if(S == NULL)
        return false;
    *x = S->data;
    return true;
}
\end{lstlisting}

\subsection{队列}

\subsubsection{队列的基本概念}

\textcolor{red}{队列(Queue)的定义}:队列是一种线性表,限定在表的一端(称为队尾,rear)进行插入,在另一端(称为队头,front)进行删除。

队列的特点是:\textbf{先进先出}(First In First Out, FIFO)。

\begin{figure}[h]
  \centering
  \begin{tikzpicture}
    \draw (0,0) rectangle (5,1);
    \draw (1,0) -- (1,1);
    \draw (2,0) -- (2,1);
    \draw (3,0) -- (3,1);
    \draw (4,0) -- (4,1);
    \node at (0.5,0.5) {1};
    \node at (1.5,0.5) {2};
    \node at (2.5,0.5) {3};
    \node at (3.5,0.5) {4};
    \node at (4.5,0.5) {5};
    \draw[->, thick] (-1,0.5) -- (0,0.5);
    \draw[->, thick] (5,0.5) -- (6,0.5);
    \node at (-1,1) {队头(front)};
    \node at (6,1) {队尾(rear)};
    \node at (-1,0) {删除};
    \node at (6,0) {插入};
  \end{tikzpicture}
  \caption{队列的结构示意图}
\end{figure}

\subsubsection{队列的基本操作}

队列的基本操作包括:

\begin{itemize}
  \item \textcolor{blue}{InitQueue(\&Q)}:初始化队列,构造一个空队列Q
  \item \textcolor{blue}{QueueEmpty(Q)}:判断队列是否为空
  \item \textcolor{blue}{EnQueue(\&Q, x)}:入队,将元素x加入队列Q
  \item \textcolor{blue}{DeQueue(\&Q, \&x)}:出队,删除队头元素并通过x返回
  \item \textcolor{blue}{GetHead(Q, \&x)}:读取队头元素,通过x返回队头元素(不改变队列)
\end{itemize}

\subsubsection{队列的顺序存储结构}

顺序队列通常使用一维数组和两个指针(队头指针front和队尾指针rear)来实现:

\begin{lstlisting}[language=C]
#define MaxSize 100
typedef struct {
    ElemType data[MaxSize];
    int front, rear;
} SqQueue;
\end{lstlisting}

简单队列在多次入队出队操作后可能会出现"假溢出"现象(队列实际未满,但队尾指针已达数组尾部)。为解决这个问题,通常采用循环队列实现。

\textcolor{red}{循环队列}的特点:
\begin{itemize}
  \item 队列元素在内存中的空间是环状的
  \item 初始时,\texttt{front = rear = 0}
  \item 入队:\texttt{rear = (rear+1) \% MaxSize}
  \item 出队:\texttt{front = (front+1) \% MaxSize}
\end{itemize}

\colorbox{yellow}{判断队列空和队列满的条件}:
\begin{itemize}
  \item 队空:\texttt{front == rear}
  \item 队满:\texttt{(rear+1) \% MaxSize == front}(牺牲一个存储单元)
\end{itemize}

\textcolor{blue}{循环队列的基本操作实现}:

1. 初始化
\begin{lstlisting}[language=C]
void InitQueue(SqQueue *Q) {
    Q->front = Q->rear = 0;
}
\end{lstlisting}

2. 判断队空
\begin{lstlisting}[language=C]
bool QueueEmpty(SqQueue Q) {
    return Q.front == Q.rear;
}
\end{lstlisting}

3. 入队操作
\begin{lstlisting}[language=C]
bool EnQueue(SqQueue *Q, ElemType x) {
    if((Q->rear+1) % MaxSize == Q->front)  // 队满
        return false;
    Q->data[Q->rear] = x;
    Q->rear = (Q->rear+1) % MaxSize;
    return true;
}
\end{lstlisting}

4. 出队操作
\begin{lstlisting}[language=C]
bool DeQueue(SqQueue *Q, ElemType *x) {
    if(Q->front == Q->rear)  // 队空
        return false;
    *x = Q->data[Q->front];
    Q->front = (Q->front+1) % MaxSize;
    return true;
}
\end{lstlisting}

5. 获取队头元素
\begin{lstlisting}[language=C]
bool GetHead(SqQueue Q, ElemType *x) {
    if(Q.front == Q.rear)  // 队空
        return false;
    *x = Q.data[Q.front];
    return true;
}
\end{lstlisting}

\subsubsection{队列的链式存储结构}

链式队列通常用带头结点的单链表实现,front指向队头结点,rear指向队尾结点:

\begin{lstlisting}[language=C]
typedef struct LinkNode {
    ElemType data;
    struct LinkNode *next;
} LinkNode;

typedef struct {
    LinkNode *front, *rear;
} LinkQueue;
\end{lstlisting}

\textcolor{blue}{链式队列的基本操作实现}:

1. 初始化(带头结点)
\begin{lstlisting}[language=C]
void InitQueue(LinkQueue *Q) {
    Q->front = Q->rear = (LinkNode *)malloc(sizeof(LinkNode));
    Q->front->next = NULL;
}
\end{lstlisting}

2. 判断队空
\begin{lstlisting}[language=C]
bool QueueEmpty(LinkQueue Q) {
    return Q.front == Q.rear;
}
\end{lstlisting}

3. 入队操作
\begin{lstlisting}[language=C]
void EnQueue(LinkQueue *Q, ElemType x) {
    LinkNode *p = (LinkNode *)malloc(sizeof(LinkNode));
    p->data = x;
    p->next = NULL;
    Q->rear->next = p;
    Q->rear = p;
}
\end{lstlisting}

4. 出队操作
\begin{lstlisting}[language=C]
bool DeQueue(LinkQueue *Q, ElemType *x) {
    if(Q->front == Q->rear)  // 队空
        return false;
    LinkNode *p = Q->front->next;
    *x = p->data;
    Q->front->next = p->next;
    if(Q->rear == p)  // 最后一个结点出队
        Q->rear = Q->front;
    free(p);
    return true;
}
\end{lstlisting}

\subsection{栈和队列的应用}

\subsubsection{栈的应用}

1. \textcolor{blue}{括号匹配问题}:检查括号是否匹配,例如 "(()[()])" 是匹配的,而 "([)]" 不匹配。

2. \textcolor{blue}{表达式求值}:利用栈实现中缀表达式转后缀表达式,再计算后缀表达式的值。

3. \textcolor{blue}{递归实现}:利用栈实现递归,避免使用系统栈。

4. \textcolor{blue}{函数调用}:程序中的函数调用使用栈来保存返回地址和局部变量。

5. \textcolor{blue}{浏览器的前进/后退功能}:利用两个栈来实现。

\begin{example}
  使用栈实现括号匹配的算法:
\begin{lstlisting}[language=C]
bool bracketMatch(char str[]) {
    Stack S;
    InitStack(&S);
    for(int i = 0; str[i] != '\0'; i++) {
        if(str[i] == '(' || str[i] == '[' || str[i] == '{') {
            Push(&S, str[i]);
        } else if(str[i] == ')' || str[i] == ']' || str[i] == '}') {
            if(StackEmpty(S))
                return false;
            char topChar;
            Pop(&S, &topChar);
            if(str[i] == ')' && topChar != '(')
                return false;
            if(str[i] == ']' && topChar != '[')
                return false;
            if(str[i] == '}' && topChar != '{')
                return false;
        }
    }
    return StackEmpty(S);
}
\end{lstlisting}
\end{example}

\subsubsection{队列的应用}

1. \textcolor{blue}{层次遍历}:广度优先搜索(BFS)算法中使用队列。

2. \textcolor{blue}{消息缓冲区}:在消息传递系统中作为消息的缓冲区。

3. \textcolor{blue}{打印机任务队列}:管理打印任务。

4. \textcolor{blue}{操作系统中的作业调度}:先来先服务(FCFS)调度算法。

5. \textcolor{blue}{计算机网络中的数据包队列}:路由器中的数据包队列。

\begin{example}
  使用队列实现二叉树层次遍历的算法:
\begin{lstlisting}[language=C]
void levelOrder(BiTree T) {
    Queue Q;
    InitQueue(&Q);
    if(T != NULL)
        EnQueue(&Q, T);
    while(!QueueEmpty(Q)) {
        BiTree p;
        DeQueue(&Q, &p);
        printf("%d ", p->data);
        if(p->lchild != NULL)
            EnQueue(&Q, p->lchild);
        if(p->rchild != NULL)
            EnQueue(&Q, p->rchild);
    }
}
\end{lstlisting}
\end{example}

\section{数组和广义表}

\subsection{数组}

\subsubsection{数组的基本概念}

\textcolor{red}{数组(Array)的定义}:数组是由n个相同类型的数据元素构成的有限序列,每个数据元素称为一个数组元素。在内存中,数组占用一块连续的存储空间。

数组的特点:
\begin{itemize}
  \item 数组中的所有元素具有相同的数据类型
  \item 数组元素在内存中按照一定的顺序连续存储
  \item 可以通过下标(索引)直接访问任何一个数组元素
\end{itemize}

一维数组可表示为 $A[0:n-1]$ 或 $A[1:n]$,其中 $A[i]$ 表示第 $i$ 个元素。

\subsubsection{数组的存储结构}

\textbf{数组的基本存储策略}:\textcolor{blue}{随机存取(Random Access)},通过公式计算元素的存储位置。

\paragraph{一维数组的存储}

假设一维数组 $A[0:n-1]$ 的起始地址为 $base$,每个元素占用 $size$ 个存储单元,则数组元素 $A[i]$ 的存储地址计算公式为:

\begin{equation}
  Loc(A[i]) = base + i \times size
\end{equation}

\begin{figure}[h]
  \centering
  \begin{tikzpicture}
    \draw (0,0) rectangle (8,1);
    \foreach \x in {1,2,...,7}
    \draw (\x,0) -- (\x,1);
    \foreach \x/\l in {0.5/A[0], 1.5/A[1], 2.5/A[2], 3.5/A[3], 4.5/A[4], 5.5/A[5], 6.5/A[6], 7.5/A[7]}
    \node at (\x,0.5) {\l};
    \draw [decorate,decoration={brace,amplitude=5pt}] (0,1.2) -- (8,1.2) node [midway,above=4pt] {内存连续存储};
    \node at (0,-0.3) {base};
    \node at (8,-0.3) {base + n*size};
  \end{tikzpicture}
  \caption{一维数组的存储结构示意图}
\end{figure}

\paragraph{多维数组的存储}

对于二维数组 $A[0:m-1][0:n-1]$,有两种常见的存储方式:

\colorbox{yellow}{行优先存储}:先存储第0行,再存储第1行,依此类推。
\begin{equation}
  Loc(A[i][j]) = base + (i \times n + j) \times size
\end{equation}

\colorbox{yellow}{列优先存储}:先存储第0列,再存储第1列,依此类推。
\begin{equation}
  Loc(A[i][j]) = base + (j \times m + i) \times size
\end{equation}

\begin{figure}[h]
  \centering
  \begin{tikzpicture}
    \draw (0,0) grid (4,3);
    \foreach \i/\j/\l in {0.5/2.5/A[0][0], 1.5/2.5/A[0][1], 2.5/2.5/A[0][2], 3.5/2.5/A[0][3],
      0.5/1.5/A[1][0], 1.5/1.5/A[1][1], 2.5/1.5/A[1][2], 3.5/1.5/A[1][3],
    0.5/0.5/A[2][0], 1.5/0.5/A[2][1], 2.5/0.5/A[2][2], 3.5/0.5/A[2][3]}
    \node at (\i,\j) {\l};

    \draw[->, thick] (5,3) to[out=0,in=90] (6,2.5);
    \node at (5,3.3) {行优先顺序};

    \draw (6,0) rectangle (14,1);
    \foreach \x in {7,8,...,13}
    \draw (\x,0) -- (\x,1);
    \foreach \x/\l in {6.5/A[0][0], 7.5/A[0][1], 8.5/A[0][2], 9.5/A[0][3], 10.5/A[1][0], 11.5/A[1][1], 12.5/A[1][2], 13.5/A[1][3]}
    \node at (\x,0.5) {\l};
  \end{tikzpicture}
  \caption{二维数组的行优先存储示意图}
\end{figure}

对于更高维的数组,存储地址计算公式可以类推得到。

\subsubsection{特殊矩阵的压缩存储}

为了节省存储空间,对于一些特殊矩阵,可以采用压缩存储的方式。

\paragraph{对称矩阵}

\textcolor{red}{对称矩阵(Symmetric Matrix)}:若对于 $n$ 阶方阵 $A$ 中的任意元素 $a_{ij} = a_{ji}$ $(i,j = 1,2,...,n)$,则称 $A$ 为对称矩阵。

对称矩阵只需存储主对角线和下(或上)三角区域的元素,共需存储 $\frac{n(n+1)}{2}$ 个元素。

\colorbox{yellow}{一维数组存储映射公式}:
\begin{equation}
  k =
  \begin{cases}
    \frac{i(i-1)}{2} + j - 1, & \text{当} i \geq j \text{(存储下三角)} \\
    \frac{j(j-1)}{2} + i - 1, & \text{当} i < j \text{(存储上三角)}
  \end{cases}
\end{equation}

\paragraph{三角矩阵}

\textcolor{red}{三角矩阵(Triangular Matrix)}:上三角矩阵的下三角部分(不含主对角线)的元素全为同一常数,下三角矩阵的上三角部分(不含主对角线)的元素全为同一常数。

\paragraph{对角矩阵}

\textcolor{red}{对角矩阵(Diagonal Matrix)}:主对角线两侧各有不超过 $s$ 条对角线上的元素可能非零,其余元素全为0的矩阵,也称为带状矩阵。

\paragraph{稀疏矩阵}

\textcolor{red}{稀疏矩阵(Sparse Matrix)}:矩阵中非零元素的个数远远少于矩阵元素总数,并且非零元素的分布没有规律。

通常采用三元组表示法存储稀疏矩阵:
\begin{lstlisting}[language=C]
typedef struct {
    int row;      // 行号
    int col;      // 列号
    ElemType val; // 值
} Triple;

typedef struct {
    Triple data[MAXSIZE+1]; // 非零元素的三元组表
    int rows, cols;         // 矩阵的行数、列数
    int nums;               // 非零元素的个数
} SparseMatrix;
\end{lstlisting}

\subsubsection{矩阵的基本运算}

\paragraph{矩阵的转置}

矩阵转置是将矩阵 $A$ 的行与列互换得到矩阵 $B$,即 $b_{ij} = a_{ji}$。

\textcolor{blue}{普通矩阵转置算法}:
\begin{lstlisting}[language=C]
void TransposeMatrix(SparseMatrix A, SparseMatrix *B) {
    B->rows = A.cols; B->cols = A.rows; B->nums = A.nums;
    if(B->nums > 0) {
        int q = 1;
        for(int col = 0; col < A.cols; col++) {
            for(int p = 1; p <= A.nums; p++) {
                if(A.data[p].col == col) {
                    B->data[q].row = A.data[p].col;
                    B->data[q].col = A.data[p].row;
                    B->data[q].val = A.data[p].val;
                    q++;
                }
            }
        }
    }
}
\end{lstlisting}

\paragraph{矩阵的乘法}

两个矩阵相乘,左矩阵的列数必须等于右矩阵的行数。

\textcolor{blue}{普通矩阵乘法算法}:
\begin{lstlisting}[language=C]
void MultiplyMatrix(int A[][N], int B[][P], int C[][P], int m, int n, int p) {
    for(int i = 0; i < m; i++) {
        for(int j = 0; j < p; j++) {
            C[i][j] = 0;
            for(int k = 0; k < n; k++) {
                C[i][j] += A[i][k] * B[k][j];
            }
        }
    }
}
\end{lstlisting}

\subsection{广义表}

\subsubsection{广义表的定义}

\textcolor{red}{广义表(Generalized List)的定义}:广义表是一种非线性的数据结构,其元素可以是原子(不可分割的数据元素)或者是广义表。

用 $LS$ 表示一个广义表,$n$ 是表的长度,则其形式定义为:
\begin{equation}
  LS = (a_1, a_2, \ldots, a_n)
\end{equation}

其中,$a_i$ 可以是原子,也可以是广义表。如果 $a_i$ 是广义表,则称为 $LS$ 的子表。

广义表的基本特点:
\begin{itemize}
  \item 广义表的元素可以是原子,也可以是广义表
  \item 广义表可以为空表,此时 $n=0$
  \item 广义表可以嵌套定义
  \item 广义表的长度是指第一层元素的个数
  \item 广义表的深度是指嵌套的最大层数
\end{itemize}

\begin{example}
  广义表示例:
  \begin{enumerate}
    \item $A = ()$:空表,长度为0,深度为1
    \item $B = (a, b, c)$:长度为3,深度为1
    \item $C = (a, (b, c), d)$:长度为3,深度为2
    \item $D = (a, (b, (c, d), e), f)$:长度为3,深度为3
    \item $E = ((a, b), (c, d), (e, f))$:长度为3,深度为2
  \end{enumerate}
\end{example}

\subsubsection{广义表的存储结构}

\paragraph{头尾链表存储表示}

每个结点包含三个域:
\begin{itemize}
  \item 标志域(Tag):区分原子结点和表结点
  \item 结点值域:
    \begin{itemize}
      \item 对于原子结点,值域存储原子的值
      \item 对于表结点,值域是指向子表的第一个元素的指针(Head)
    \end{itemize}
  \item 指针域(Tail):指向当前结点所在表中的下一个元素
\end{itemize}

\begin{lstlisting}[language=C]
typedef enum {ATOM, LIST} ElemTag;  // ATOM=0:原子,LIST=1:子表
typedef struct GLNode {
    ElemTag tag;              // 标志域
    union {                   // 值域
        AtomType atom;        // tag=ATOM时,存储原子值
        struct {              // tag=LIST时
            struct GLNode *hp; // 指向表头
            struct GLNode *tp; // 指向表尾
        } ptr;
    };
} GLNode, *GList;
\end{lstlisting}

\paragraph{扩展线性链表存储表示}

每个结点有三个域:
\begin{itemize}
  \item 标志域(Tag):区分原子结点和表结点
  \item 数据域:
    \begin{itemize}
      \item 对于原子结点,数据域存储原子的值
      \item 对于表结点,数据域是指向子表的指针
    \end{itemize}
  \item 指针域(Next):指向下一个元素
\end{itemize}

\begin{lstlisting}[language=C]
typedef enum {ATOM, LIST} ElemTag;  // ATOM=0:原子,LIST=1:子表
typedef struct GLNode {
    ElemTag tag;                   // 标志域
    union {                        // 值域
        AtomType atom;             // tag=ATOM时,存储原子值
        struct GLNode *sublist;    // tag=LIST时,指向子表
    } val;
    struct GLNode *next;           // 指向下一个元素
} GLNode, *GList;
\end{lstlisting}

\subsubsection{广义表的基本操作}

\textcolor{blue}{广义表的基本操作}包括:
\begin{itemize}
  \item 求广义表的长度(第一层元素的个数)
  \item 求广义表的深度(嵌套的最大层数)
  \item 复制广义表
  \item 创建广义表
\end{itemize}

求广义表的深度

\begin{lstlisting}[language=C]
int GListDepth(GList L) {
    if(!L) return 1;  // 空表深度为1

    if(L->tag == ATOM) return 0;  // 原子深度为0

    int max = 0;
    int dep;
    GList p = L->val.sublist;  // p指向子表

    while(p) {
        dep = GListDepth(p);
        if(dep > max) max = dep;
        p = p->next;
    }

    return max + 1;  // 当前层深度+1
}
\end{lstlisting}

复制广义表
\begin{lstlisting}[language=C]
void CopyGList(GList *T, GList L) {
    *T = NULL;
    if(!L) return;

    *T = (GList)malloc(sizeof(GLNode));
    (*T)->tag = L->tag;

    if(L->tag == ATOM) {
        (*T)->val.atom = L->val.atom;
    } else {
        CopyGList(&((*T)->val.sublist), L->val.sublist);
    }

    CopyGList(&((*T)->next), L->next);
}
\end{lstlisting}

\end{document}