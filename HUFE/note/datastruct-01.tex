\documentclass{../../note}

\usepackage{amsthm}
\usepackage{pgfplots}
\pgfplotsset{compat=1.18}
\newtheorem{example}{Example}
\usepackage{xcolor} % For colored text
\usepackage{tikz}
\usetikzlibrary{shapes,arrows,positioning,fit,calc,matrix}

\title{数据结构 01}
\author{isomo}

\begin{document}

\maketitle

考纲内容:

\begin{itemize}
  \item \textbf{绪论: }
    数据、数据元素、数据结构、数据类型、抽象数据类型的概念,数据的逻辑结构和存储结构,算法、算法描述和算法分析的概念。

  \item \textbf{线性表: }
    线性表的定义及其抽象数据类型描述,顺序表的逻辑结构定义及其基本运算,链表的逻辑结构及其基本操作。
\end{itemize}

\section{绪论}

% Adding a figure showing the relationship between key concepts
\begin{figure}[ht]
  \centering
  \begin{tikzpicture}[
      node distance=1.5cm and 2cm,
      box/.style={draw, rounded corners, minimum width=2.5cm, minimum height=1cm, align=center, fill=lightgray!20},
      arrow/.style={->,>=stealth, thick}
    ]
    % Main concepts
    \node[box, fill=red!10] (data) {数据\\Data};
    \node[box, fill=red!10, below=of data] (dataelem) {数据元素\\Data Element};
    \node[box, fill=red!10, below=of dataelem] (datast) {数据结构\\Data Structure};
    \node[box, fill=blue!10, below left=of datast] (datatype) {数据类型\\Data Type};
    \node[box, fill=blue!10, below right=of datast] (ADT) {抽象数据类型\\ADT};

    % Arrows
    \draw[arrow] (data) -- (dataelem) node[midway, right] {组成};
    \draw[arrow] (dataelem) -- (datast) node[midway, right] {形成};
    \draw[arrow] (datast) -- (datatype);
    \draw[arrow] (datatype) -- (ADT) node[midway, above] {抽象化};
  \end{tikzpicture}
  \caption{数据结构基本概念关系图}
\end{figure}

\subsection{基本概念}

\subsubsection{数据}
\textcolor{red}{数据(Data)}是对客观事物的符号表示,是信息的载体,是计算机程序加工的对象。数据是计算机科学研究的基本对象,也是计算机处理的对象。

\subsubsection{数据元素}
\textcolor{red}{数据元素(Data Element)}是数据的基本单位,是数据集合中的一个客体,也称为记录。数据元素通常由若干个数据项组成,每个数据项是数据元素的一个最小单位,也称为字段。

\subsubsection{数据结构}
\textcolor{red}{数据结构(Data Structure)}是指相互之间存在一种或多种特定关系的数据元素的集合。一个数据结构包括三个方面:
\begin{enumerate}
  \item 数据的逻辑结构
  \item 数据的存储结构
  \item 数据的操作
\end{enumerate}

\subsubsection{数据类型}
\textcolor{red}{数据类型(Data Type)}是指一组性质相同的值的集合及定义在此集合上的一组操作的总称。数据类型可分为基本数据类型(如整型、实型、字符型等)和结构化数据类型(如数组、记录、文件等)。

\subsubsection{抽象数据类型}
\textcolor{red}{抽象数据类型(Abstract Data Type, ADT)}是指一个数学模型以及定义在该模型上的一组操作。抽象数据类型把数据类型的逻辑特性与其实现细节分离,使用ADT来描述数据结构时只需关注其逻辑特性,而不必关心其具体实现。

\subsection{数据的结构}

% Adding a figure for data structures
\begin{figure}[ht]
  \centering
  \begin{tikzpicture}[
      node distance=1.5cm,
      box/.style={draw, rounded corners, minimum width=2.5cm, minimum height=0.8cm, align=center},
      arrow/.style={->,>=stealth, thick}
    ]
    % Main division
    \node[box, fill=yellow!20] (datastruct) {数据结构};

    \node[box, fill=blue!10, below left=1cm and 2cm of datastruct] (logical) {逻辑结构};
    \node[box, fill=red!10, below right=1cm and 2cm of datastruct] (storage) {存储结构};

    % Logical structures
    \node[box, below left=0.8cm and 0.3cm of logical] (set) {集合结构};
    \node[box, below=0.8cm of logical] (linear) {线性结构};
    \node[box, below right=0.8cm and 0.3cm of logical] (tree) {树形结构};
    \node[box, below=0.8cm of tree] (graph) {图形结构};

    % Storage structures
    \node[box, below left=0.8cm and 0.3cm of storage] (seq) {顺序存储};
    \node[box, below=0.8cm of storage] (linked) {链式存储};
    \node[box, below right=0.8cm and 0.3cm of storage] (indexed) {索引存储};
    \node[box, below=0.8cm of indexed] (hash) {散列存储};

    % Arrows
    \draw[arrow] (datastruct) -- (logical);
    \draw[arrow] (datastruct) -- (storage);
    \draw[arrow] (logical) -- (set);
    \draw[arrow] (logical) -- (linear);
    \draw[arrow] (logical) -- (tree);
    \draw[arrow] (logical) -- (graph);
    \draw[arrow] (storage) -- (seq);
    \draw[arrow] (storage) -- (linked);
    \draw[arrow] (storage) -- (indexed);
    \draw[arrow] (storage) -- (hash);
  \end{tikzpicture}
  \caption{数据结构的分类}
\end{figure}

\subsubsection{数据的逻辑结构}
\textcolor{red}{逻辑结构(Logical Structure)}是指数据元素之间的逻辑关系,与数据的存储无关。主要分为以下四类:

% Visualizing logical structures
\begin{figure}[ht]
  \centering
  \begin{tikzpicture}[scale=0.8]
    % Set structure
    \begin{scope}[local bounding box=set, shift={(-5,0)}]
      \draw (0,0) circle (1.5);
      \foreach \i in {0,72,...,288}
      \filldraw (\i:1) circle (0.1);
      \node[below=0.5cm] at (0,-1.5) {集合结构};
    \end{scope}

    % Linear structure
    \begin{scope}[local bounding box=linear, shift={(0,0)}]
      \foreach \i in {0,...,4} {
        \filldraw (\i,0) circle (0.1);
        \ifnum\i<4
        \draw[->] (\i+0.1,0) -- (\i+0.9,0);
        \fi
      }
      \node[below=0.5cm] at (2,-0.1) {线性结构};
    \end{scope}

    % Tree structure
    \begin{scope}[local bounding box=tree, shift={(-3,-4)}]
      \filldraw (0,0) circle (0.1) node (root) {};
      \foreach \i in {-1,0,1} {
        \filldraw (\i,-1) circle (0.1) node (level1_\i) {};
        \draw[->] (root) -- (level1_\i);
      }
      \foreach \i/\pos in {1/-1.5, 2/-1, 3/-0.5, 4/0.5, 5/1, 6/1.5} {
        \filldraw (\pos,-2) circle (0.1) node (level2_\i) {};
      }
      \draw[->] (level1_-1) -- (level2_1);
      \draw[->] (level1_-1) -- (level2_2);
      \draw[->] (level1_0) -- (level2_3);
      \draw[->] (level1_0) -- (level2_4);
      \draw[->] (level1_1) -- (level2_5);
      \draw[->] (level1_1) -- (level2_6);
      \node[below=0.5cm] at (0,-2.2) {树形结构};
    \end{scope}

    % Graph structure
    \begin{scope}[local bounding box=graph, shift={(3,-4)}]
      \foreach \i/\x/\y in {1/0/0, 2/2/0, 3/3/1.5, 4/1/2, 5/-1/1.5} {
        \filldraw (\x,\y) circle (0.1) node (n\i) {};
      }
      \draw[->] (n1) -- (n2);
      \draw[->] (n2) -- (n3);
      \draw[->] (n3) -- (n4);
      \draw[->] (n4) -- (n5);
      \draw[->] (n5) -- (n1);
      \draw[->] (n1) -- (n4);
      \draw[->] (n2) -- (n5);
      \node[below=0.5cm] at (1,0) {图形结构};
    \end{scope}
  \end{tikzpicture}
  \caption{数据的逻辑结构示意图}
\end{figure}

\begin{enumerate}
  \item 集合结构:数据元素之间除了同属一个集合外,无其他关系
  \item 线性结构:数据元素之间存在一对一的线性关系
  \item 树形结构:数据元素之间存在一对多的层次关系
  \item 图形结构:数据元素之间存在多对多的任意关系
\end{enumerate}

\subsubsection{数据的存储结构}
\textcolor{red}{存储结构(Storage Structure)}是指数据结构在计算机中的表示,也称为物理结构。主要分为以下四类:

% Visualizing storage structures
\begin{figure}[ht]
  \centering
  \begin{tikzpicture}[scale=0.8]
    % Sequential storage
    \begin{scope}[local bounding box=seq, shift={(-4,0)}]
      \foreach \i in {0,...,4} {
        \draw (\i,0) rectangle ++(1,1);
        \node at (\i+0.5,0.5) {$a_\i$};
      }
      \node[below=0.5cm] at (2.5,0) {顺序存储};
    \end{scope}

    % Linked storage
    \begin{scope}[local bounding box=linked, shift={(3,-2)}]
      % First define all coordinates
      \foreach \i/\x/\y in {0/0/0, 1/2/0, 2/4/0, 3/3/1.5, 4/1/1.5} {
        \coordinate (n\i) at (\x,\y);
      }

      % Then draw the nodes and connections
      \foreach \i/\x/\y in {0/0/0, 1/2/0, 2/4/0, 3/3/1.5, 4/1/1.5} {
        \draw (\x,\y) rectangle ++(1,1);
        \draw (\x+0.5,\y) -- ++(0,1);
        \node at (\x+0.25,\y+0.5) {$a_\i$};

        \ifnum\i<4
        \node at (\x+0.75,\y+0.5) {$\bullet$};
        \draw[->, thick] (\x+0.75,\y+0.5) to[out=0,in=180] ([shift={(0,0.5)}]n\number\numexpr\i+1);
        \else
        \node at (\x+0.75,\y+0.5) {$\emptyset$};
        \fi
      }
      \node[below=0.5cm] at (2.5,0) {链式存储};
    \end{scope}
  \end{tikzpicture}
  \caption{顺序存储与链式存储结构示意图}
\end{figure}

\begin{enumerate}
  \item 顺序存储:将数据元素存储在地址连续的存储单元中
  \item 链式存储:将数据元素存储在任意的存储单元中,通过指针来表示元素之间的逻辑关系
  \item 索引存储:在存储元素信息的同时,还建立附加的索引表
  \item 散列存储:根据数据元素的关键字直接计算出该元素的存储地址
\end{enumerate}

\subsection{算法}

\subsubsection{算法的概念}
\textcolor{red}{算法(Algorithm)}是对特定问题求解步骤的一种描述,是一组有穷的、确定的、可行的指令序列。算法具有五个基本特性:

% Algorithm characteristics visualization
\begin{figure}[ht]
  \centering
  \begin{tikzpicture}
    \node[draw, rounded corners, fill=yellow!20, minimum width=3cm, minimum height=1cm] (algo) at (0,0) {算法};

    \foreach \i/\name/\angle in {
      1/有穷性/90,
      2/确定性/162,
      3/可行性/234,
      4/输入/306,
      5/输出/18
    } {
      \node[draw, rounded corners, fill=blue!10, minimum width=2cm, minimum height=0.8cm] (char\i) at (\angle:3.5cm) {\name};
      \draw[->, thick] (algo) -- (char\i);
    }
  \end{tikzpicture}
  \caption{算法的五个基本特性}
\end{figure}

\begin{enumerate}
  \item 有穷性:算法必须在有限的步骤后终止
  \item 确定性:每一步骤都有明确的定义
  \item 可行性:每一步操作都是可以实现的
  \item 输入:有零个或多个输入
  \item 输出:有一个或多个输出
\end{enumerate}

\subsubsection{算法描述}
算法描述是对算法进行表达的方式,常用的描述方法包括:
\begin{enumerate}
  \item 自然语言描述
  \item 流程图描述
  \item 伪代码描述
  \item 程序语言描述
\end{enumerate}

\subsubsection{算法分析}
\textcolor{red}{算法分析(Algorithm Analysis)}是指对算法的执行效率和所需资源进行评估和预测。主要包括:

% Time complexity visualization
\begin{figure}[ht]
  \centering
  \begin{tikzpicture}
    \begin{axis}[
        width=10cm,
        height=6cm,
        xlabel={问题规模 $n$},
        ylabel={执行时间},
        domain=1:10,
        samples=100,
        legend pos=north west,
        legend style={font=\small},
        grid=both,
        ymin=0, ymax=100
      ]
      \addplot[color=blue, thick] {1};
      \addlegendentry{$O(1)$}

      \addplot[color=green!60!black, thick] {ln(x)/ln(2)};
      \addlegendentry{$O(\log n)$}

      \addplot[color=orange, thick] {x};
      \addlegendentry{$O(n)$}

      \addplot[color=red, thick] {x*ln(x)/ln(2)};
      \addlegendentry{$O(n\log n)$}

      \addplot[color=purple, thick] {x^2};
      \addlegendentry{$O(n^2)$}

      \addplot[color=brown, thick] {2^x};
      \addlegendentry{$O(2^n)$}
    \end{axis}
  \end{tikzpicture}
  \caption{常见时间复杂度增长率比较}
\end{figure}

\begin{enumerate}
  \item 时间复杂度:评估算法执行所需的时间
  \item 空间复杂度:评估算法执行所需的存储空间
\end{enumerate}

时间复杂度通常用大O符号表示,常见的时间复杂度有:$O(1)$、$O(\log n)$、$O(n)$、$O(n\log n)$、$O(n^2)$、$O(n^3)$、$O(2^n)$等。

\section{线性表}

% Concept relationship diagram
\begin{figure}[ht]
  \centering
  \begin{tikzpicture}[
      node distance=1.5cm and 2cm,
      box/.style={draw, rounded corners, minimum width=2.5cm, minimum height=1cm, align=center, fill=lightgray!20},
      arrow/.style={->,>=stealth, thick}
    ]
    % Main concepts
    \node[box, fill=yellow!20] (linearlist) {线性表\\Linear List};

    % Storage implementations
    \node[box, fill=red!10, below left=of linearlist] (seqlist) {顺序表\\Sequential List};
    \node[box, fill=red!10, below right=of linearlist] (linkedlist) {链表\\Linked List};

    % Linked list variants
    \node[box, fill=blue!10, below left=1cm and 0cm of linkedlist] (singlelist) {单链表};
    \node[box, fill=blue!10, below=1cm of linkedlist] (doublelist) {双链表};
    \node[box, fill=blue!10, below right=1cm and 0cm of linkedlist] (circularlist) {循环链表};

    % Arrows
    \draw[arrow] (linearlist) -- (seqlist);
    \draw[arrow] (linearlist) -- (linkedlist);
    \draw[arrow] (linkedlist) -- (singlelist);
    \draw[arrow] (linkedlist) -- (doublelist);
    \draw[arrow] (linkedlist) -- (circularlist);
  \end{tikzpicture}
  \caption{线性表的分类及关系}
\end{figure}

\subsection{线性表的定义}

\textcolor{red}{线性表(Linear List)}是具有相同数据类型的n个数据元素的有限序列。其中:
\begin{itemize}
  \item 数据元素的个数n定义为表的长度
  \item 当n=0时称为空表
  \item 表中任一元素所在位置后有且仅有一个直接后继元素
  \item 表中除第一个元素外,任一元素有且仅有一个直接前驱元素
\end{itemize}

\begin{figure}[ht]
  \centering
  \begin{tikzpicture}
    % Draw elements
    \foreach \i/
  \item in {1/a_1, 2/a_2, 3/a_3, 4/a_4, 5/a_5, 6/a_n} {
      \ifnum\i=6
      \node[draw, circle, minimum size=1.2cm] (e\i) at (\i*1.5-1.5,0) {$
    \item$};
      \draw[dotted, thick] (e5) -- (e6);
      \else
      \node[draw, circle, minimum size=1.2cm] (e\i) at (\i*1.5-1.5,0) {$
    \item$};
      \fi
    }

    % Connect elements with arrows
    \foreach \i in {1,...,5} {
      \pgfmathtruncatemacro{\j}{\i+1}
      \ifnum\i=5
      % Don't draw arrow to the last node as it's not directly connected
      \else
      \draw[->, thick] (e\i) -- (e\j);
      \fi
    }

    % Add annotations
    \node[above=0.5cm] at (e1) {第一个元素};
    \node[below=0.5cm] at (e1) {直接后继};
    \node[below=0.5cm] at (e2) {直接前驱};
    \node[above=0.5cm] at (e6) {最后一个元素};
  \end{tikzpicture}
  \caption{线性表的逻辑结构示意图}
\end{figure}

\subsection{线性表的抽象数据类型描述}

\textcolor{red}{线性表的ADT}可以用以下方式表述:
\begin{itemize}
  \item \textbf{数据对象}: 线性表是n个数据元素的有限序列
  \item \textbf{数据关系}: 除最后一个元素外,每个元素有且仅有一个直接后继;除第一个元素外,每个元素有且仅有一个直接前驱
  \item \textbf{基本操作}: 初始化、销毁、插入、删除、查找、更新等
\end{itemize}

\begin{figure}[ht]
  \centering
  \begin{tikzpicture}[
      operation/.style={draw, rounded corners, fill=green!10, minimum width=2cm, minimum height=0.7cm}
    ]
    % Draw ADT box
    \draw[rounded corners, fill=yellow!10] (0,0) rectangle (10,-5) node[pos=.5] {};
    \node[font=\bfseries] at (5,0.5) {线性表ADT};

    % Basic operations
    \node[operation] at (2.5,-1) {InitList()};
    \node[operation] at (7.5,-1) {DestroyList()};

    \node[operation] at (2.5,-2) {ListInsert()};
    \node[operation] at (7.5,-2) {ListDelete()};

    \node[operation] at (2.5,-3) {GetElem()};
    \node[operation] at (7.5,-3) {LocateElem()};

    \node[operation] at (2.5,-4) {ListLength()};
    \node[operation] at (7.5,-4) {ListEmpty()};
  \end{tikzpicture}
  \caption{线性表的抽象数据类型}
\end{figure}

\subsection{顺序表}

\subsubsection{顺序表的定义}
\textcolor{red}{顺序表(Sequential List)}是用一组地址连续的存储单元依次存储线性表的数据元素,从而使得逻辑上相邻的两个元素在物理位置上也相邻。

\begin{figure}[ht]
  \centering
  \begin{tikzpicture}
    % 顺序表的内存表示
    \foreach \i in {0,...,9} {
      \draw (\i,0) rectangle ++(1,1);
      \ifnum\i<5
      \node at (\i+0.5,0.5) {$a_{\i+1}$};
      \else
      \node at (\i+0.5,0.5) {};
      \fi
    }

    % 标注
    \draw[decorate, decoration={brace, amplitude=10pt}] (0,1.2) -- (5,1.2)
    node[midway, above=10pt] {已存储元素(length=5)};
    \draw[decorate, decoration={brace, amplitude=10pt}] (5,1.2) -- (10,1.2)
    node[midway, above=10pt] {未使用空间};

    % 索引标注
    \foreach \i in {0,...,9} {
      \node[below=0.3cm] at (\i+0.5, 0) {$[\i]$};
    }

    % 数组名和基地址
    \node[below=0.8cm] at (0.5, 0) {base};
  \end{tikzpicture}
  \caption{顺序表的存储结构}
\end{figure}

\subsubsection{顺序表的特点}
\begin{itemize}
  \item 优点:随机访问(通过位置直接计算地址),存储密度高
  \item 缺点:插入删除需要移动元素,大小固定难以扩展
\end{itemize}

\subsubsection{顺序表的基本运算}
顺序表的基本运算时间复杂度:
\begin{center}
  \begin{tabular}{|l|c|c|}
    \hline
    \textbf{操作} & \textbf{最好情况} & \textbf{最坏情况} \\
    \hline
    按值查找 & $O(1)$ & $O(n)$ \\
    \hline
    按位查找 & $O(1)$ & $O(1)$ \\
    \hline
    插入操作 & $O(1)$ & $O(n)$ \\
    \hline
    删除操作 & $O(1)$ & $O(n)$ \\
    \hline
  \end{tabular}
\end{center}

\begin{figure}[ht]
  \centering
  \begin{tikzpicture}
    % 初始顺序表
    \foreach \i in {0,...,6} {
      \draw (\i,3) rectangle ++(1,1);
      \node at (\i+0.5,3.5) {$a_{\i+1}$};
      \node[above=0.2cm] at (\i+0.5, 4) {$[\i]$};
    }

    % 插入操作(在位置2插入x)
    \draw[->, thick, red] (3.5,3) -- (3.5,1.5);
    \node[right] at (3.5,0.5) {元素后移};

    \foreach \i in {0,...,7} {
      \draw (\i,1) rectangle ++(1,1);
      \ifcase\i
      \node at (\i+0.5,1.5) {$a_1$};
      \or
      \node at (\i+0.5,1.5) {$a_2$};
      \or
      \node at (\i+0.5,1.5) {$x$};
      \or
      \node at (\i+0.5,1.5) {$a_3$};
      \or
      \node at (\i+0.5,1.5) {$a_4$};
      \or
      \node at (\i+0.5,1.5) {$a_5$};
      \or
      \node at (\i+0.5,1.5) {$a_6$};
      \or
      \node at (\i+0.5,1.5) {$a_7$};
      \fi
      \node[above=0.2cm] at (\i+0.5, 2) {$[\i]$};
    }

    % 插入位置标注
    \node[below=0.3cm] at (2.5, 1) {插入位置};
    \draw[->] (2.5,0.7) -- (2.5,1);
  \end{tikzpicture}
  \caption{顺序表的插入操作示意图}
\end{figure}

\subsection{链表}

\subsubsection{链表的定义}
\textcolor{red}{链表(Linked List)}是用一组任意的存储单元来依次存储线性表中的数据元素,元素的存储顺序不必和逻辑顺序一致,通过指针来指示元素的逻辑关系。

\begin{figure}[ht]
  \centering
  \begin{tikzpicture}
    % 头指针
    \node[draw, minimum width=1cm, minimum height=0.8cm] (head) at (0,0) {head};

    % 节点
    \foreach \i/\x in {1/2, 2/4, 3/6, 4/8, 5/10} {
      % 数据域和指针域
      \draw (\x,0) rectangle ++(1,0.8);
      \draw (\x+1,0) rectangle ++(0.5,0.8);
      \node at (\x+0.5,0.4) {$a_\i$};

      \ifnum\i<5
      \node at (\x+1.25,0.4) {$\bullet$};
      \draw[->] (\x+1.25,0.4) -- ++(0.75,0);
      \else
      \node at (\x+1.25,0.4) {$\emptyset$};
      \fi
    }

    % 连接头指针到第一个节点
    \draw[->] (head) -- (2,0.4);

    % 标注
    \node[above=0.7cm] at (2.5,0) {数据域};
    \node[below=0.3cm] at (3.25,0) {指针域};
    \node[below=0.3cm] at (0,0) {头指针};
    \node[below=0.3cm] at (10.75,0) {NULL};
  \end{tikzpicture}
  \caption{单链表结构示意图}
\end{figure}

\subsubsection{链表的特点}
\begin{itemize}
  \item 优点:插入删除不需要移动元素,只需修改指针,大小可动态调整
  \item 缺点:不支持随机访问,需要从头开始查找,存储密度低(需要额外的指针空间)
\end{itemize}

\subsubsection{链表的分类}
链表根据结构可分为:
\begin{enumerate}
  \item 单链表:每个结点只有一个指针域,指向后继结点
  \item 双链表:每个结点有两个指针域,分别指向前驱和后继结点
  \item 循环链表:尾节点的指针指向头节点形成环状结构
\end{enumerate}

\begin{figure}[ht]
  \centering
  \begin{tikzpicture}[
      node distance=2cm,
      list/.style={rectangle split, rectangle split parts=2, draw, rectangle split horizontal}
    ]
    % 双链表
    \node[above] at (4.5,2) {双链表};
    \foreach \i/\data in {1/a_1, 2/a_2, 3/a_3, 4/a_4} {
      \node[list] (node\i) at (\i*2,1) {\nodepart{one}$\bullet$\nodepart{two}$\data$};
    }
    \node[list] (node5) at (10,1) {\nodepart{one}$\bullet$\nodepart{two}$a_5$};

    \foreach \i in {1,2,3,4} {
      \pgfmathtruncatemacro{\j}{\i+1}
      \draw[->] ([xshift=0.3cm]node\i.east) -- ([xshift=-0.3cm]node\j.west);
      \draw[->] ([xshift=-0.3cm]node\j.west) -- ([xshift=0.3cm]node\i.east);
    }

    % 循环链表
    \node[above] at (4.5,-1) {循环链表};
    \foreach \i/\x in {1/1, 2/3, 3/5, 4/7, 5/9} {
      \draw (\x,-2) rectangle ++(1,0.8);
      \draw (\x+1,-2) rectangle ++(0.5,0.8);
      \node at (\x+0.5,-1.6) {$a_\i$};
      \node at (\x+1.25,-1.6) {$\bullet$};
    }

    \foreach \i/\x in {1/1, 2/3, 3/5, 4/7} {
      \pgfmathtruncatemacro{\j}{\x+2}
      \draw[->] (\x+1.25,-1.6) -- (\j,-1.6);
    }

    % 循环箭头
    \draw[->] (10.25,-1.6) to[out=0,in=0] (1,-2);
    % \draw (10.25,-1) to[out=0,in=0] (1,-1.6);
  \end{tikzpicture}
  \caption{双链表与循环链表示意图}
\end{figure}

\subsubsection{链表的基本操作}
链表的基本操作时间复杂度:
\begin{center}
  \begin{tabular}{|l|c|c|}
    \hline
    \textbf{操作} & \textbf{顺序表} & \textbf{链表} \\
    \hline
    按值查找 & $O(n)$ & $O(n)$ \\
    \hline
    按位查找 & $O(1)$ & $O(n)$ \\
    \hline
    插入操作 & $O(n)$ & $O(1)$ \\
    \hline
    删除操作 & $O(n)$ & $O(1)$ \\
    \hline
  \end{tabular}
\end{center}

\begin{figure}[ht]
  \centering
  \begin{tikzpicture}
    % 初始链表
    \node[draw, minimum width=1cm, minimum height=0.8cm] (head) at (0,2) {head};

    \foreach \i/\x in {1/1, 2/3, 3/5, 4/7} {
      \draw (\x+1,2) rectangle ++(1,0.8);
      \draw (\x+2,2) rectangle ++(0.5,0.8);
      \node at (\x+1.5,2.4) {$a_\i$};

      \ifnum\i<4
      \node at (\x+2.25,2.4) {$\bullet$};
      \draw[->] (\x+2.25,2.4) -- ++(0.75,0);
      \else
      \node at (\x+2.25,2.4) {$\emptyset$};
      \fi
    }

    \draw[->] (head) -- (2,2.4);

    % 插入操作
    \node at (4.5,1) {在$a_2$和$a_3$之间插入$x$};

    \node[draw, minimum width=1cm, minimum height=0.8cm] (head2) at (0,0) {head};

    \foreach \i/\x/\data in {1/1/a_1, 2/3/a_2, 3/5/x, 4/7/a_3, 5/9/a_4} {
      \draw (\x+1,0) rectangle ++(1,0.8);
      \draw (\x+2,0) rectangle ++(0.5,0.8);
      \node at (\x+1.5,0.4) {$\data$};

      \ifnum\i<5
      \node at (\x+2.25,0.4) {$\bullet$};
      \draw[->] (\x+2.25,0.4) -- ++(0.75,0);
      \else
      \node at (\x+2.25,0.4) {$\emptyset$};
      \fi
    }

    \draw[->] (head2) -- (2,0.4);

    % 新节点高亮
    \draw[red, thick] (6,0) rectangle ++(1.5,0.8);

    % 指针修改
    \draw[->, red, thick] (5.25,2.4) to[out=-45,in=45] (6.25,0.4);
    \draw[->, red, thick] (5.25,0.4) -- (6,0.4);
  \end{tikzpicture}
  \caption{链表的插入操作示意图}
\end{figure}

\end{document}