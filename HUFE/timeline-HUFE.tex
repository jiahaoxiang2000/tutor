\documentclass{../note}

\title{课程时间安排 HUFE}
\author{isomo}

\begin{document}

\maketitle

\section{考纲内容}

\subsection{考核目标}
\begin{enumerate}
  \item 考核学生对各种数据结构的基本概念与基本原理的理解和掌握,以及运用数据结构知识分析问题和解决问题的能力。
  \item 考核学生对数据库系统的基本概念与基本原理的理解和掌握,以及运用数据库设计方法分析问题和解决问题的能力。
\end{enumerate}

\subsection{考核内容}
\subsubsection{数据结构}
\begin{enumerate}
  \item \textbf{绪论}\\
    数据、数据元素、数据结构、数据类型、抽象数据类型的概念,数据的逻辑结构和存储结构,算法、算法描述和算法分析的概念。

  \item \textbf{线性表}\\
    线性表的定义及其抽象数据类型描述,顺序表的逻辑结构定义及其基本运算,链表的逻辑结构及其基本操作。

  \item \textbf{栈和队列}\\
    栈的结构特性、基本操作及在顺序存储结构和链式存储结构上基本运算的实现,队列的结构特性、基本操作及在顺序存储结构和链式存储结构上基本运算的实现,栈和队列的基本应用。

  \item \textbf{数组和广义表}\\
    数组的基本概念和存储结构,广义表的定义和存储结构。

  \item \textbf{树和二叉树}\\
    树的基本概念,二叉树的概念、性质和存储结构,二叉树的遍历,线索二叉树,哈夫曼树。

  \item \textbf{图}\\
    图的基本概念,图的存储结构(邻接矩阵、邻接表、十字链表和邻接多重表),图的遍历,生成树和最小生成树,最短路径。

  \item \textbf{查找}\\
    查找的基本概念,线性表的查找,二叉排序树,哈希表的查找。

  \item \textbf{内排序}\\
    排序的基本概念,各种排序(插入排序、交换排序、选择排序、归并排序和基数排序)的基本思想和算法分析。
\end{enumerate}

\subsubsection{数据库原理}
\begin{enumerate}
  \item \textbf{绪论}\\
    数据库的4个基本概念,数据管理技术的产生和发展,数据建模、概念模型和数据模型的三要素,数据库系统的三级模式结构,数据库的两级映像与数据独立性,数据库系统的组成。

  \item \textbf{关系模型}\\
    关系模型的数据结构及形式化定义,关系操作,关系完整性,关系代数(传统的集合运算、专门的关系运算)。

  \item \textbf{关系数据库标准语言SQL}\\
    数据定义、数据查询、数据更新、空值处理、视图。

  \item \textbf{数据库安全性}\\
    数据库安全性概述,数据库安全性控制。

  \item \textbf{数据库完整性}\\
    数据库完整性概述,实体完整性,参照完整性,用户定义完整性,完整性约束命名子句。

  \item \textbf{关系数据理论}\\
    关系数据库规范化理论的基本概念,函数依赖的定义和函数依赖的公理系统,第一/二/三范式和BC范式,关系模式的分解。

  \item \textbf{数据库设计}\\
    数据库设计的基本步骤及各阶段的主要任务,E-R模型及用E-R模型进行概念结构设计,逻辑结构设计。

  \item \textbf{数据库恢复和并发控制}\\
    事务的基本概念,故障的种类,恢复的实现技术,恢复策略及具有检查点的恢复技术;并发控制的基本概念。
\end{enumerate}

\section{课程时间安排}

基于考纲内容和各章节的难度,我们将24小时的辅导时间按照以下方式进行安排:

\subsection{总体时间分配}
\begin{itemize}
  \item 数据结构:14小时(较大比重,难度较高)
  \item 数据库原理:10小时
  \item 总计:24小时
\end{itemize}

\subsection{详细时间安排}

\subsubsection{数据结构部分 (14小时)}

\begin{tabular}{|l|c|p{8cm}|}
  \hline
  \textbf{章节内容} & \textbf{时间分配} & \textbf{教学要点} \\
  \hline
  1. 绪论 & 1小时 &
  数据结构的基本概念与分类,算法分析方法 \\
  \hline
  2. 线性表 & 2小时 &
  顺序表和链表的实现与基本操作,应用场景分析,重点讲解单链表、双链表和循环链表的操作 \\
  \hline
  3. 栈和队列 & 2小时 &
  栈和队列的实现方法,顺序和链式存储结构的比较,经典应用问题(表达式求值、递归消除等) \\
  \hline
  4. 数组和广义表 & 1小时 &
  多维数组的存储,广义表的概念与基本操作 \\
  \hline
  5. 树和二叉树 & 3小时 &
  二叉树的性质、遍历算法与应用,线索二叉树的构造与使用,哈夫曼树的构建和编码应用 \\
  \hline
  6. 图 & 2小时 &
  图的基本概念与存储结构,图的遍历算法,最小生成树(Prim和Kruskal算法),最短路径(Dijkstra算法) \\
  \hline
  7. 查找 & 1.5小时 &
  顺序查找、折半查找,二叉排序树的构建与操作,哈希表及其处理冲突的方法 \\
  \hline
  8. 内排序 & 1.5小时 &
  各类排序算法的原理与实现,时间复杂度和空间复杂度分析与比较 \\
  \hline
\end{tabular}

\subsubsection{数据库原理部分 (10小时)}

\begin{tabular}{|l|c|p{8cm}|}
  \hline
  \textbf{章节内容} & \textbf{时间分配} & \textbf{教学要点} \\
  \hline
  1. 绪论 & 0.5小时 &
  数据库基本概念,三级模式结构与两级映像 \\
  \hline
  2. 关系模型 & 1.5小时 &
  关系数据模型的基本概念,关系代数操作(选择、投影、连接等) \\
  \hline
  3. SQL语言 & 2.5小时 &
  DDL、DML、DCL语句,复杂查询(子查询、连接查询、集合操作等),视图的使用 \\
  \hline
  4. 数据库安全性 & 0.5小时 &
  安全机制实现方法,访问控制技术 \\
  \hline
  5. 数据库完整性 & 1小时 &
  实体完整性、参照完整性和用户自定义完整性约束的实现 \\
  \hline
  6. 关系数据理论 & 2小时 &
  函数依赖理论,各种范式的定义与转换,模式分解 \\
  \hline
  7. 数据库设计 & 1.5小时 &
  E-R模型设计,概念模型到逻辑模型的转换,规范化过程 \\
  \hline
  8. 数据库恢复和并发控制 & 0.5小时 &
  事务的ACID属性,锁机制,并发控制方法,恢复技术 \\
  \hline
\end{tabular}

\end{document}