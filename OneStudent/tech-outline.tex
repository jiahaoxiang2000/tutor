\documentclass{article}
% Use with [notoc] option to hide table of contents
\usepackage[notoc]{../typesetting/styles/note-zh}
% Default shows table of contents
% \usepackage{../styles/note-zh}
\usepackage{bookmark}

\title{Tech-Outline 一生一系统 项目教学大纲}
\author{isomo}

\begin{document}

\maketitle

\section{课程基本信息}

以下我们对于题目有两个方向,一个是服务器密码机,二是故障攻击解算密钥(我们不熟悉,待研究)

\subsection{私有云服务器密码机设计与实现}

项目名称:私有云服务器密码机设计与实现

参考行业标准:GM/T 0030-2014 《服务器密码机技术规范》,GM/T 0104-2021《云服务器密码机技术规范》。

\section{课程教学目标}

私有云环境中,数据加密、身份认证、密钥管理等安全问题日益突出,同时面临着内部威胁、外部攻击、监管合规等多重挑战。服务器密码机作为安全基础设施的核心组件,通过提供高强度的密码算法支持、安全的密钥生成与管理、可靠的加解密服务等功能,为私有云环境构建了坚实的安全屏障。为私有云计算安全提出一些解决方案。要做到有技术创新、有应用意义。获得省级软件应用作品竞赛三等奖或以上。

本课程教学需培养以下四方面能力:加强学生对加密技术基础知识的掌握和理解;培养学生的设计思维和逻辑思考能力;增强学生的团队合作与沟通能力;提高学生对信息安全的重视和意识课程;通过项目制教学方法,理解密码学基础知识,设计和实现加密系统,指导学生进行发明专利、软件著作权、互联网+等竞赛的计划书撰写。

通过完成这个项目,学生将学会设计和实现一个安全加密系统的步骤和流程。他们也将获得实际操作密码学知识和技能的机会。此外,这个项目还能培养学生团队合作、沟通和问题解决的能力。 

\section{教学内容与学时分配}

主要参考书本《密码工程学》、《轻量级分组密码》以及密码机相关知识。了解目前以及未来私有云面临的安全问题,熟悉私有云内部通信协议,了解加密技术,在此基础上搭建一个服务器密码机(软件或者硬件),为私有云提供一个安全方案。获得发明专利、软件著作权。参加互联网+、挑战杯等竞赛并获奖,并实现成果产业应用。

\begin{table}[htbp]
\centering
\begin{tabular}{|p{2cm}|p{1.5cm}|p{3.5cm}|p{4cm}|p{3cm}|}
\hline
\textbf{教学时间} & \textbf{学时} & \textbf{教学内容} & \textbf{阶段性教学目标} & \textbf{教学形式(在线或面授)} \\
\hline
2025年6月28日 & 2学时 & 私有云环境面临的安全问题 & 了解私有云安全问题 & 每周五晚7:30-10:00腾讯会议在线教学 \\
\hline
2025年7月5日 & 2学时 & 密码学 & 了解密码学的基本算法 & \\
\hline
2025年7月12日 & 2学时 & 对学生的项目选题进行指导 & 确定项目题目 & \\
\hline
2025年7月19日--2025年8月9日 & 每周2学时,共计8学时 & 指导学生完成私有云服务器密码机总体结构设计和模块设计 & 确定项目任务,确定项目的模块结构 & \\
\hline
2025年8月16日--2025年8月30日 & 每周2学时,共计6学时 & 指导学生学习相关基础知识 & 掌握项目搭建的基本知识 & \\
\hline
2025年9月20日--2025年11月29日 & 每两周2学时,共计12学时 & 指导学生完成私有云服务器密码机框架功能的编写 & 私有云服务器密码机框架基本功能完成 & \\
\hline
2025年12月13日 & 2学时 & 指导学生的发明专利申请书 & 每一个项目提交一份发明专利申请(或软件系统) & \\
\hline
2025年12月27日 & 2学时 & 指导学生进行互联网+、各类省级竞赛作品计划书以及PPT分工 & 了解互联网+比赛的具体要求,合理分工 & \\
\hline
2026年2月21日--2026年4月3日 & 每两周2学时,共计8学时 & 指导学生完成互联网+比赛、各类省级竞赛的计划书以及PPT的书写 & 完成计划书以及PPT的书写 & \\
\hline
2026年4月17日 & 2学时 & 提交十佳原创参赛作品至学院.同时准备互联网+竞赛、各类省级竞赛相关材料提交 & & \\
\hline
\end{tabular}
\end{table}

\end{document}
