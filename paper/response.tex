\documentclass{beamer}
\usepackage{../typesetting/styles/slide-zh}

\setCJKmainfont{STSong}
\setCJKsansfont{Kaiti SC}  % Better sans-serif font for Chinese
\setCJKmonofont{STFangsong}  % Better monospace font for Chinese

% Document information
\title{\red{重生之我是审稿人}}
\subtitle{换位解读-审稿意见}
\author{向嘉豪}
% \institute{机构名称}
\date{\today}

\begin{document}

% Title frame
\begin{frame}
  \titlepage
\end{frame}

% Outline frame
\begin{frame}{大纲}
  \tableofcontents
\end{frame}

% Introduction
\section{今天要写个审稿意见}
\begin{frame}{今天要写个审稿意见}
  \large
  当审稿人第一次打开你的稿件时,内心的小剧场可能是这样的:

  \vspace{0.5em}
  \red{“又来一篇新稿子,今天能不能早点下班就看它了!”}

  \vspace{0.5em}
  \blue{“标题看起来还挺有意思,希望不是标题党。”}

  \vspace{0.5em}
  \green{“摘要先来一眼,嗯……这到底在说啥?我是不是还没喝咖啡?”}

  \vspace{0.5em}
  \red{“公式好多,图片也不少,作者是不是想用视觉冲击让我晕头转向?”}

  \vspace{0.5em}
  \blue{“先别急着下结论,翻到结论部分看看作者到底想表达什么。”}

  \vspace{0.5em}
  \green{“参考文献这么多,看来作者很努力地‘致敬’了同行。”}
\end{frame}

\begin{frame}{今天要写个审稿意见:启示}
  \large
  总之,审稿人面对稿件时,既有期待,也有疑惑,偶尔还会有点小吐槽。\par
  \vspace{1em}
  \red{理解他们的心理活动,有助于我们更好地“对症下药”,写出让审稿人会心一笑的回复!}\par
  \vspace{1em}
  所以,写回复时不妨多站在审稿人的角度,提前“剧透”他们的疑问和关注点,\blue{用心沟通,效果加倍!}
\end{frame}

% Special boxes
\section{看不太懂稿件怎么写}
\begin{frame}{看不太懂稿件怎么写}
  \large
  有时候,审稿人面对一篇“高深莫测”的稿件,内心可能是这样的:

  \vspace{0.5em}
  \red{“这是什么新名词?我是不是打开了假论文?”}

  \vspace{0.5em}
  \blue{“作者的逻辑跳跃有点大,感觉像在坐过山车。”}

  \vspace{0.5em}
  \green{“是不是我太久没学习了,还是这领域真的太前沿?”}

  \vspace{0.5em}
  \red{“看了三遍摘要,还是没明白核心创新点在哪……”}

  \vspace{0.5em}
  \blue{“要不要去查查参考文献,看看别人是不是也这么写?”}

  \vspace{1em}
  面对“看不太懂”的稿件,审稿人往往会\blue{反复阅读}、\green{查资料},甚至怀疑人生。此时,\red{清晰的表达}和\blue{友好的结构},就是作者送给审稿人的最大温暖!
\end{frame}

\begin{frame}{看不太懂稿件怎么写:意见与回复}
  \large
  当审稿人真的“看不懂”时,评论可能会这样写:

  \vspace{0.5em}
  \red{“本稿部分内容表述不够清晰,建议作者进一步解释相关概念和方法。”}

  \vspace{0.5em}
  \blue{“论文结构略显跳跃,建议理顺逻辑关系,便于读者理解。”}

  \vspace{1em}
  这时,作为作者,我们可以这样机智回应:

  \vspace{0.5em}
  \green{“感谢审稿人的宝贵意见!我们已对相关部分进行了详细补充和澄清,并优化了论文结构,使逻辑更加清晰。”}

  \vspace{1em}
  \blue{只要态度诚恳,表达清楚,‘看不懂’也能变‘秒懂’!}
\end{frame}

% Math example
\section{同行稿件要好好看看}
\begin{frame}{同行稿件要好好看看}
  \large
  当审稿人发现稿件和自己研究方向“撞车”时,内心戏格外丰富:

  \vspace{0.5em}
  \red{“咦,这不是我去年刚发的那个点子吗?”}

  \vspace{0.5em}
  \blue{“作者引用了我的论文,心情不错,先给个好评!”}

  \vspace{0.5em}
  \green{“不过,和我的方法比起来,这里是不是还可以更完善?”}

  \vspace{0.5em}
  \red{“同行的工作要仔细看,既要公正,也要专业,不能‘护犊子’。”}

  \vspace{1em}
  面对同行稿件,审稿人会\blue{格外认真},既要肯定创新,也会“鸡蛋里挑骨头”。所以,\red{充分引用相关工作}、\blue{诚恳承认不足},是赢得同行好感的秘诀!
\end{frame}

\begin{frame}{同行稿件要好好看看:意见与回复}
  \large
  审稿人常见评论:

  \vspace{0.5em}
  \red{“建议作者补充与已有工作的对比,突出本稿创新点。”}

  \vspace{0.5em}
  \blue{“部分方法与文献[XX]类似,请详细说明区别与改进。”}

  \vspace{1em}
  作者可以这样回应:

  \vspace{0.5em}
  \green{“感谢审稿人的建议!我们已补充与相关工作的详细对比,并在X节突出本稿的创新与改进之处。”}

  \vspace{1em}
  \blue{同行是最严格的考官,也是最懂你的观众。用心对比,坦诚交流,同行也会为你点赞!}
\end{frame}

% Thank you slide
\begin{frame}
  \centering
  \LARGE 谢谢

\end{frame}

\end{document}
