\documentclass{../note}

\title{课程时间安排 FACSU}
\author{isomo}

\begin{document}

\maketitle

\section{考纲内容}
\begin{itemize}
  \item \textbf{第一章 Java 语言概述}
    \begin{itemize}
      \item 知识要点:Java 语言的产生、应用前景和特点;Java 虚拟机及 Java 运行系统;Java 语言和 C++语言的相同及不同之处;熟练掌握 Java 应用程序的编辑、编译和运行过程。
      \item 考试要求:
        \begin{enumerate}
          \item 了解:Java 语言的产生、应用前景和特点
          \item 了解:Java 虚拟机及 Java 运行系统
          \item 掌握:Java 语言和 C++语言的相同及不同之处
          \item 掌握:Java 应用程序的编写、编译和运行过程
        \end{enumerate}
    \end{itemize}

  \item \textbf{第二章 Java 语言基础}
    \begin{itemize}
      \item 知识要点:Java 语言的数据类型;变量和常量;正确书写表达式;数组;流程控制。
      \item 考试要求:
        \begin{enumerate}
          \item 了解:数据类型的转换(自动类型转换和强制类型转换);运算符的优先级和结合性
          \item 掌握:Java 语言各种数据类型
          \item 掌握:Java 语言算术运算符、关系运算符、逻辑运算符、位运算符和复合赋值运算符的功能及使用
          \item 掌握:Java 语言变量、常量的使用及其运算操作
          \item 掌握:Java 语言流程控制语句的功能及使用
          \item 掌握:Java 数组的定义;数组的初始化和数组的应用;二维数组的应用
        \end{enumerate}
    \end{itemize}

  \item \textbf{第三章 面向对象编程}
    \begin{itemize}
      \item 知识要点:面向对象的基本概念;面向对象的软件开发过程。
      \item 考试要求:
        \begin{enumerate}
          \item 了解:面向对象的概念
          \item 掌握:类的创建与使用
          \item 掌握:方法的定义和使用
          \item 掌握:对象的基本操作方式
          \item 掌握:构造方法的定义和使用
          \item 掌握:this 关键字和 static 关键字的使用
          \item 理解:成员变量和局部变量的区别
        \end{enumerate}
    \end{itemize}

  \item \textbf{第四章 面向对象的特性}
    \begin{itemize}
      \item 知识要点:掌握面向对象的三大特性。
      \item 考试要求:
        \begin{enumerate}
          \item 理解:封装的概念
          \item 理解:继承的概念
          \item 理解:多态的概念
          \item 掌握:final 关键字的使用
          \item 掌握:Lambda 表示式的使用
        \end{enumerate}
    \end{itemize}

  \item \textbf{第五章 抽象类和接口}
    \begin{itemize}
      \item 知识要点:抽象类与接口的基本概念以及实际应用。
      \item 考试要求:
        \begin{enumerate}
          \item 掌握:抽象类和接口的使用
          \item 掌握:Java 中的内部类
          \item 了解:单例模式
          \item 了解:模板设计方法
        \end{enumerate}
    \end{itemize}

  \item \textbf{第六章 Java 异常}
    \begin{itemize}
      \item 知识要点:Java 异常的基本概念;Java 异常处理机制;自定义 Java 异常类的应用。
      \item 考试要求:
        \begin{enumerate}
          \item 理解:异常的概念
          \item 掌握:异常的处理机制
          \item 掌握:自定义异常的使用
        \end{enumerate}
    \end{itemize}

  \item \textbf{第七章 Java 常用类}
    \begin{itemize}
      \item 知识要点:应用 Java 语言的工具类库。
      \item 考试要求:
        \begin{enumerate}
          \item 掌握:字符串相关类的使用
          \item 掌握:System 类与 Runtime 类的使用
          \item 掌握:Math 类与 Random 类的使用
          \item 掌握:日期类的使用
        \end{enumerate}
    \end{itemize}

  \item \textbf{第八章 集合框架}
    \begin{itemize}
      \item 知识要点:应用 Java 语言的集合框架解决具体问题。
      \item 考试要求:
        \begin{enumerate}
          \item 掌握:List、Map、Set 集合的使用
          \item 掌握:集合遍历的方法
          \item 掌握:泛型的使用
          \item 掌握:集合工具类的使用
          \item 掌握:Stream API 的使用
        \end{enumerate}
    \end{itemize}

  \item \textbf{第九章 Java IO}
    \begin{itemize}
      \item 知识要点:Java 输入输出与文件处理。
      \item 考试要求:
        \begin{enumerate}
          \item 掌握:File 类及其用法
          \item 掌握:操作字节流和字符流读写文件
          \item 了解:其他 IO 流
          \item 了解:NIO 的概念及其用法
          \item 了解:常见字符编码
        \end{enumerate}
    \end{itemize}

  \item \textbf{第十章 图形用户界面}
    \begin{itemize}
      \item 知识要点:Java 的 Swing 组件、容器、布局管理器的概念;图形界面上的事件响应。
      \item 考试要求:
        \begin{enumerate}
          \item 了解:AWT 组件和 Swing 组件的联系和区别
          \item 掌握:常用的 Swing 组件的使用
          \item 理解:常用的窗体和布局管理器
          \item 掌握:事件处理机制
        \end{enumerate}
    \end{itemize}

  \item \textbf{第十一章 Java 多线程}
    \begin{itemize}
      \item 知识要点:多线程的基本概念;创建和启动线程;线程的生命周期;多线程同步问题;多线程通信;线程池的概念。
      \item 考试要求:
        \begin{enumerate}
          \item 了解:进程和线程的区别
          \item 掌握:创建线程的方法
          \item 理解:线程的生命周期及其状态转换
          \item 掌握:多线程的同步
          \item 掌握:多线程之间的通信
          \item 了解:线程池的使用
        \end{enumerate}
    \end{itemize}

  \item \textbf{第十二章 Java 网络编程}
    \begin{itemize}
      \item 知识要点:网络协议;使用 Java 开发网络程序。
      \item 考试要求:
        \begin{enumerate}
          \item 了解:网络通信协议
          \item 了解:UDP 通信
          \item 了解:TCP 通信
          \item 掌握:网络程序的开发
        \end{enumerate}
    \end{itemize}

  \item \textbf{第十三章 JDBC 编程}
    \begin{itemize}
      \item 知识要点:数据库基本概念;JDBC 原理;应用 JDBC 接口操作数据库。
      \item 考试要求:
        \begin{enumerate}
          \item 了解:JDBC 原理
          \item 掌握:Connection 接口、Statement 接口、ResultSet 接口、PreparedStatement 接口的使用
          \item 掌握:使用 JDBC 操作数据库
        \end{enumerate}
    \end{itemize}
\end{itemize}

\section{课程时间安排}

% now we have 24 hours time to tutor this course, How to arrange the time? Consider the difficulty of each chapter and the exam requirements on different levels for the check points.

\subsection{按章节分配时间}
基于各章节的难度和考纲要求,24小时教学时间安排如下:

\begin{tabular}{|l|l|c|p{8cm}|}
  \hline
  \textbf{章节} & \textbf{时间分配} & \textbf{小时} & \textbf{重点内容} \\
  \hline
  第一章 Java 语言概述 & 第1小时 & 1 & Java基础知识,JVM概念,与C++对比 \\
  \hline
  第二章 Java 语言基础 & 第2-4小时 & 3 & 数据类型,运算符,流程控制,数组(重点掌握基础语法) \\
  \hline
  第三章 面向对象编程 & 第5-7小时 & 3 & 类的创建与使用,方法,构造函数,this与static关键字 \\
  \hline
  第四章 面向对象的特性 & 第8-9小时 & 2 & 封装,继承,多态,final关键字,Lambda表达式 \\
  \hline
  第五章 抽象类和接口 & 第10-11小时 & 2 & 抽象类,接口实现,内部类,设计模式简介 \\
  \hline
  第六章 Java 异常 & 第12小时 & 1 & 异常处理机制,try-catch-finally,自定义异常 \\
  \hline
  第七章 Java 常用类 & 第13-14小时 & 2 & 字符串处理,System类,Math类,日期时间API \\
  \hline
  第八章 集合框架 & 第15-17小时 & 3 & 集合类的使用,泛型,Stream API(重点掌握) \\
  \hline
  第九章 Java IO & 第18-19小时 & 2 & 文件操作,流的概念与应用 \\
  \hline
  第十章 图形用户界面 & 第20小时 & 1 & Swing基础组件,事件处理 \\
  \hline
  第十一章 Java 多线程 & 第21-22小时 & 2 & 线程创建,同步,通信(重要且较难) \\
  \hline
  第十二章 Java 网络编程 & 第23小时 & 1 & 网络通信基础,Socket编程 \\
  \hline
  第十三章 JDBC 编程 & 第24小时 & 1 & 数据库连接,SQL操作 \\
  \hline
\end{tabular}
\end{document}